\documentclass[../../Heldenanleitung2]{subfiles}
\begin{document}

\chapter{Regeln}

Wird ein Attribut mit seiner Abkürzung erwähnt, so ist damit immer das Ergebnis des entsprechenden Würfelwurfs gemeint zuzüglich eines eventuellen Bonus oder Malus den das Attribut hat. Ist die Rede von einem maximalem Attribut (z.B. Max Bew im Falle der Rüstungsklasse), so ist die größte Zahl die der entsprechende Attributswürfel würfeln kann plus ein eventueller Bonus oder Malus gemeint. Generell zählt ein Wurf kritisch sobald die höchste Zahl des Würfels geworfen wurde. In diesem Fall wird erneut gewürfelt und das neue Ergebnis hinzu addiert. Der neue Wurf kann erneut kritisch sein. Würfe bei denen ein erneutes Würfeln um den Würfelwurf zu erhöhen sinnlos ist, können nicht kritisch sein (z.B. der Narbenwurf).

\section{Probe}
\begin{table}
\caption{Richtwerte für die SK der Proben}
\centering
\begin{tabular}{|c|c|}
\hline
Schwierigkeit & Würfelwert \\
\hline
sehr leicht & 4\\
leicht & 6\\
normal & 9\\
schwierig & 11\\
sehr schwierig & 14\\
extrem schwierig & 20\\ \hline
\end{tabular}
\end{table}
Zum Absolvieren einer Probe wird mit dem Attributswürfel und einem Probenwürfel gewürfelt. Zu dem Würfelergebnis werden alle eventuelle Boni und Mali gezählt. Der Spielleiter entscheidet die Schwierigkeit der Probe und somit den Wert der gewürfelt werden muss. Die Standardschwierigkeit für Proben ist 9.

Bei vergleichenden Proben wie z.B. Schleichen gegen Bemerken werden beide Probenwerte miteinander verglichen. Bei Gleichstand ist zu Gunsten des Spielers zu entscheiden. Es kann jedoch sein, dass einer der beiden Würfe einen Bonus bekommt. Ist es zum Beispiel sehr dunkel und gleichzeitig laut, so bekommt der Bemerkenwurf einen Malus. Ist der Schleichende aber mit lauten Glocken unterwegs, so ist diesem ein Malus zu geben.

Bei einem Probenwurf können Steigerungen erreicht werden. Für je 4 die der PW über dem zu erreichenden Ziel liegt, ist eine Steigerung erreicht. Steigerungen charakterisieren besonders gelungene Aktionen, so können zusätzliche positive Effekte neben dem eigentlichen Ziel auftreten.

Besondere Proben sind Sprinten und Reagieren. Jede Steigerung der Probe Sprinten erhöht die Bewegungsweite um $1$\,m und jede Steigerung der Probe Ausweichen erhöht die BK um 1.

\subsection{Notwendiges Wissen}
Ist eine Probe mit einem * markiert, so ist es notwendig, dass der Spieler die Probe mindestens einmal gesteigert haben muss, um sie benutzen zu können. Ist es notwendig dass der Spieler trotzdem auf die Probe wirft, so bekommt er einen Malus von 4.

\section{Leveln von Proben}
Für einen Fp, kann der Probenwürfel einer Probe vergrößert werden. Pro Level kann eine Probe immer nur einmal gesteigert werden. Eine Probe kann maximal Spielerlevel/2 (aufgerundet) mal gesteigert werden. Jede Probe startet mit einem W4 als Probenwürfel.

\section{Kampf}
Kämpfe finden in einem rundenbasierten System statt. Eine Kampfrunde entspricht dem Geschehen von 5 Sekunden.

\subsection{Initiative}
Beginnt eine Kampfsituation, so werfen alle beteiligten Individuen auf Initiative. Die Ergebnisse legen die Reihenfolge fest in der sie dran sind, das Individuum mit dem höchsten Wert beginnt. Bei Gleichstand hat das Individuum mit der höheren maximalen Initiative Vorrang(wenn keine kritischen Würfe möglich wären). Ist auch das identisch so gibt es ein Stechen.

\subsection{Aktionsphase}
Wenn ein Individuum an der Reihe ist, kann es sich um seinen Laufenwert bewegen und hat eine Aktion sowie eine Bonusaktion zur Verfügung. Die Aktion kann verwendet werden um beispielsweise einen Angriff durchzuführen. Bei diversen anderen Tätigkeiten ist vom Spielleiter zu entscheiden, ob die Tätigkeit eine Aktion kostet oder wie das rufen kurzer Sätze nur eine Bonusaktion. Es sind auch freie Aktionen möglich für kurze Handlungen wie umschauen. Ein Spieler kann seine Aktionen halten, um zu einem späteren Zeitpunkt zu agieren. Dazu beschreibt er ein Szenario bei dessen Eintreten er die gewünschte Aktion durchführen will. Ist der Spieler erneut an der Reihe, ohne dass das genannte Szenario eingetreten ist, verfällt die gehaltene Aktion.

\subsection{Aktionen}
\renewcommand{\arraystretch}{1.5}
\begin{table}
\caption{Aktionstabelle}
\label{tab:Aktionen}
\begin{tabular}{|p{0.2\textwidth}|p{0.2\textwidth}|p{0.6\textwidth}|}
\hline
Name & Kosten & Effekt\\
\hline
Laufen & Laufenaktion & Laufen Meter bewegen \\
Angreifen & Aktion & Führt Angriff durch \\
Schleichen & Laufenaktion & Laufen/4 Meter schleichend fortbewegen\\
Sprinten & Volle Aktion & 3*Laufen Meter in nahezu gerader Linie bewegen\\
Kleine Aktion & Bonusaktion & Rufen, werfen, fangen o.ä.\\
Waffe ziehen & Aktion / Bonusaktion & Wenn Probe auf Fingerfertigkeit $\geq 18$ dann schnelle Aktion, sonst Standardaktion\\
Waffe wegstecken & Aktion / Bonusaktion & Wenn Probe auf Fingerfertigkeit $\geq 18$ dann schnelle Aktion, sonst Standardaktion\\
Waffe fallen lassen & freie Aktion & Waffe fällt auf den Boden\\
Defensive Kampfhaltung & keine & -3 auf Angriffswürfe und +2 auf RK. Es kann nur in eine defensive Kampfhaltung gewechselt werden, wenn in dieser Runde noch nicht angegriffen wurde.\\
Einschüchtern & Aktion & Einschüchternprobe gegen Willenskraftprobe des Ziels. Wenn erfolgreich, Ziel wird verängstigt. Für jede Steigerung wird ein weiteres Verängstigtlevel hinzugefügt.\\
Führen/Begeistern & Aktion & Führen/Begeistern-Probe. Wenn erfolgreich, alle Verbündeten in Hörreichweite bekommen Moral. Für jede Steigerung wird ein weiteres Morallevel hinzugefügt.\\
Verspotten & Aktion & Verspottenprobe gegen Willenskraftprobe des Ziels. Wenn erfolgreich, Ziel wird verspottet. Für jede Steigerung wird ein weiteres Verspottenlevel hinzugefügt.\\
\hline
\end{tabular}
\end{table}
In Tabelle \ref{tab:Aktionen} sind mögliche Aktionen aufgezählt, die in einer Kampfsituation durchgeführt werden könnten. Es können alle erdenklichen Aktionen durchgeführt werden die in einer Situation möglich sein könnten. Der Spieler kann immer beschreiben was er tun möchte und welches Ziel so erreicht werden soll. Der Spielleiter entscheidet dann, ob es möglich ist und welche Probe gegebenenfalls notwendig ist.

\subsection{Angriffe}
Bei einem Angriff würfelt der Spieler auf eine Waffenprobe, dabei erhält er eventuelle Angriffsboni durch die Waffe. Ist der Probenwert größer oder gleich der Rüstungsklasse des Ziels, so trifft der Angriff. Bei einem Treffer wird der Schaden, wie bei der Waffe angegeben, ausgewürfelt und von den Lp des Ziels abgezogen. Eingeschränkte Bewegungsfreiheit kann sowohl Angriffswurf als auch Rüstungsklasse reduzieren. Ist einem Angriffsziel nicht klar, dass es angegriffen wird oder ist es bewegungsunfähig, so wird die BK auf 4 reduziert.

Besitzt eine Waffe x\% Rüstungsdurchdringung und schlägt ein Angriff fehl, hätte aber getroffen, wenn das Ziel keine Rüstung getragen hätte, so wird x\% vom Schaden angerichtet.

\subsection{Gezielte Angriffe und Kampfmanöver}
Ein Spieler kann jeder Zeit die Art seines Angriffs spezialisieren um eine bestimmte Wirkung zu erreichen. Je nach dem wie der Spieler seinen Angriff deklariert, legt der Spielleiter eine Erschwernis und einen Bonus fest. Zum Beispiel kann der Spieler sagen, er zielt mit einem Schwertstoß direkt in das offene Visier des Gegners. Der Spielleiter könnte dann eine Erschwernis von 5 festlegen und wenn der Charakter trifft sind alle Schadenswürfel automatisch kritisch. Der Spieler könnte auch sagen, versucht mit seinem Angriff den Gegner zu entwaffnen, je nach Können des Gegners und Bewaffnung wäre eine Erschwernis festzulegen. Wenn der Angriff dann erfolgreich ist, würde der Angriff keinen Schaden verursachen und stattdessen würde der Angreifer die Waffen weg wirbeln oder abnehmen.

In einem weiterem Beispiel kann der Charakter mit einem Angriff auf die Beine des Gegners zielen, um ihn bewegungsunfähig zu machen. In diesem Fall sollte der Spielleiter einen passenden Malus auf den Angriffswurf anwenden und bei Erfolg hat der Angriff neben Schaden auch den gewünschten Effekt. Als anderes Beispiel könnte ein Spieler sagen, dass sein Charakter die Axt in einem großen Bogen schwingt, um den Gegner neben seinem primären Ziel ebenfalls zu treffen. Dann könnte der Spielleiter den Angriff um 3 erschweren und wenn das primäre Ziel getroffen wird, hat das zweite Ziel auch die Möglichkeit getroffen zu werden.

\subsection{Kampf mit zwei Waffen}
Führt ein Charakter zwei Waffen, so kann er mit einer einzigen Angriffsaktion mit beiden Waffen angreifen. Der erste Angriffswurf erhält einen Malus von 2 und der zweite Angriff von 4.

\subsection{Fernkampfangriffe}
Fernkampfangriffe haben drei Reichweiten normale Reichweite, mittlere Reichweite und weite Reichweite. Befindet sich das Ziel eines Fernkampfangriffs innerhalb der normalen Reichweite, wird ganz normal auf Angriff geworfen. Befindet sich das Ziel in mittlere Reichweite, erhält der Angriffswurf einen Malus von 2 und in weiter Reichweite von 4. Ist bei einer Fernkampfwaffe nur eine Reichweite angegeben, ist die mittlere Reichweite die 1,5 fache Entfernung und die weite die doppelte Reichweite.

\subsection{Deckung}
Ein Individuum befindet sich in Deckung, wenn zwischen ihm und dem Angreifer Objekte oder andere Personen stehen. Sind mindestens 50\% des Ziels aus der Sicht des Angreifers verdeckt, hält der Angreifer einen Malus von 2 auf den Angriff. Sind mindestens 75\% verdeck steigt der Malus auf 5 und ist das Ziel vollständig verdeckt, scheitert der Angriff automatisch. Wird ein Fernkampfangriff auf ein Ziel in Deckung durchgeführt und schlägt fehl, kann das Hindernis welche die Deckung bereitstellt getroffen werden (nach Ermessen des Spielleiters). Insbesondere kann bei solch einem Angriff ein eigener Verbündeter getroffen werden.

\subsection{Gelegenheitsangriffe}
Standardmäßig hat jeder Charakter einen Gelegenheitsangriff pro Kampfrunde. Wird ein Gelegenheitsangriff ausgelöst, so wird wie bei einem Angriff verfahren. Gelegenheitsangriffe werden ausgelöst, wenn
\begin{itemize}
	\item ein Charakter sich in Nahkampfreichweite eines anderen befindet und sich von diesem entfernt ohne ihn anzugreifen.
	\item ein Charakter einen Nahkampfangriff gegen einen Gegner durchführt, der eine deutlich höhere Nahkampfreichweite besitzt und dabei der Angriffswurf kleiner der BK des Ziels ist.
	\item ein Charakter aufwendige Aktionen (z.B. verarzten) vollführt und sich dabei in Nahkampfreichweite eines Gegners befindet.
	\item ein Charakter einen Fernkampfangriff in der Nahkampfreichweite eines Gegners durchführt.
\end{itemize}

\subsection{Schleichangriffe}
Führt ein Charakter einen Nahkampfangriff gegen ein Ziel durch, das sich dem Angreifer nicht bewusst ist, so sind die Schadenswürfel der Waffe automatisch kritisch (das betrifft keine Extrawürfel für Schleichangriffe).

\subsection{Flankieren}
Ist eine Gruppe von Individuen in einem Nahkampf verwickelt, so erhalten alle Individuen der Partei, die sich in Unterzahl befindet, einen Malus von 2 auf die BK.



\subsection{Steigerungen bei Angriffen}
Wird bei einem Angriff eine oder mehrere Steigerungen erreicht, so steigt der Schaden um 1 pro Steigerung.

\subsection{Überraschungsrunden}
Wenn der Kampf für einen Teilnehmer überraschend beginnt, so setzt dieser eine Runde lang aus.

\subsection{Waffen}
Waffen besitzen eine Reichweite und können einen Malus auf Schleichen geben. Der Wert Max Ang, gibt an wie oft eine Angriffsaktion pro Runde durchgeführt werden kann, wenn diese Waffe benutzt wird. Eine Waffe gibt einen Bonus auf den Angriffswurf und die Rüstungsklasse. Es darf immer nur ein Rüstungsklassenbonus von Waffen verwendet werden. Trägt also ein Charakter zwei Schwerter, so wird nur der Rüstungsklassenbonus eines Schwertes benutzt. Der Rüstungsklassenbonus kann nur gegen Nahkampfangriffe verwendet werden, außer es handelt sich um einen Schild.

\subsection{Beispielwaffen}

\begin{tabular}{|p{0.33\textwidth}|p{0.33\textwidth}|p{0.33\textwidth}|}
\hline
\textbf{Faust} & Reichweite: 0,5\,m & Max Ang: 2 \\
\hline
Angriff: +0 & Schaden: Stä-1 & Schleichen: 0\\
\hline
\multicolumn{3}{|p{0.99\textwidth}|}{Schlägt der Angriff fehl und hat das Ziel eine Waffe ausgerüstet, so wird ein Gelegenheitsangriff provoziert.} \\
\hline
\end{tabular}
\newline \newline\newline
\begin{tabular}{|p{0.33\textwidth}|p{0.33\textwidth}|p{0.33\textwidth}|}
\hline
\textbf{Schlagring} & Reichweite: 0,5\,m &  Max Ang: 2\\
\hline
Angriff: +0 & Schaden: Stä+1 & Schleichen: 0\\
\hline
\multicolumn{3}{|p{0.99\textwidth}|}{Andere Waffen können ausgerüstet werden ohne die Schlagringe unauszurüsten, es gibt lediglich einen Malus von 1 auf den Angriffswurf.} \\
\hline
\end{tabular}
\newline \newline\newline
\begin{tabular}{|p{0.33\textwidth}|p{0.33\textwidth}|p{0.33\textwidth}|}
\hline
\textbf{Schwert (1H)} & Reichweite: 1,2\,m & Max Ang: 1\\
\hline
Angriff: +2 & Schaden: Stä+2 & Schleichen: 0\\
\hline
\multicolumn{3}{|p{0.99\textwidth}|}{Ausfallschritt: -2RK, +2 auf Angriffswurf, Reichweite +0,5\,m.} \\
\hline
\end{tabular}
\newline \newline\newline
\begin{tabular}{|p{0.33\textwidth}|p{0.33\textwidth}|p{0.33\textwidth}|}
\hline
\textbf{Langschwert (2H)} & Reichweite: 1,5\,m & Max Ang: 1 \\
\hline
Angriff: +3 & Schaden: Stä+1W6 & Schleichen: -2\\
\hline
\multicolumn{3}{|p{0.99\textwidth}|}{Ausfallschritt: -2RK, +2 auf Angriffswurf, Reichweite +0,5\,m.} \\
\hline
\end{tabular}
\newline \newline\newline
\begin{tabular}{|p{0.33\textwidth}|p{0.33\textwidth}|p{0.33\textwidth}|}
\hline
\textbf{Bidenhänder (2H)} & Reichweite: 2\,m & Max Ang: 1\\
\hline
Angriff: +1 & Schaden: Stä+1W10 & Schleichen: -3\\
\hline
\multicolumn{3}{|p{0.99\textwidth}|}{Ausfallschritt: -2RK, +2 auf Angriffswurf, Reichweite +0,5\,m.} \\
\hline
\end{tabular}
\newline \newline\newline
\begin{tabular}{|p{0.33\textwidth}|p{0.33\textwidth}|p{0.33\textwidth}|}
\hline
\textbf{Streitaxt (1H)} & Reichweite: 1\,m & Max Ang: 1\\
\hline
Angriff: +1 & Schaden: Stä+1W6 & Schleichen: -1\\
\hline
\multicolumn{3}{|p{0.99\textwidth}|}{50\% Rüstungsdurchdringung.} \\
\hline
\end{tabular}
\newline \newline\newline
\begin{tabular}{|p{0.33\textwidth}|p{0.33\textwidth}|p{0.33\textwidth}|}
\hline
\textbf{Dänenaxt (2H)} & Reichweite: 1,5\,m & Max Ang: 1\\
\hline
Angriff: +1 & Schaden: Stä+1W8+1 & Schleichen: -2\\
\hline
\multicolumn{3}{|p{0.99\textwidth}|}{Stoßen (Standardaktion): Kann einen Schildträger stoßen bei gewonnener Stemmen/Schieben Probe fällt Ziel zu Boden. 50\% Rüstungsdurchdringung.} \\
\hline
\end{tabular}
\newline \newline\newline
\begin{tabular}{|p{0.33\textwidth}|p{0.33\textwidth}|p{0.33\textwidth}|}
\hline
\textbf{Speer (1H)} & Reichweite: 2,5\,m|Max Stä/2 & Max Ang: 1\\
\hline
Angriff: +3 & Schaden: Stä+1W6 & Schleichen: -5\\
\hline
\multicolumn{3}{|p{0.99\textwidth}|}{} \\
\hline
\end{tabular}
\newline \newline\newline
\begin{tabular}{|p{0.33\textwidth}|p{0.33\textwidth}|p{0.33\textwidth}|}
\hline
\textbf{Langspeer (2H)} & Reichweite: 3,5\,m|Max Stä/4 & Max Ang: 1\\
\hline
Angriff: +3 & Schaden: Stä+1W8 & Schleichen: -7\\
\hline
\multicolumn{3}{|p{0.99\textwidth}|}{} \\
\hline
\end{tabular}
\newline \newline\newline
\begin{tabular}{|p{0.33\textwidth}|p{0.33\textwidth}|p{0.33\textwidth}|}
\hline
\textbf{Mordaxt (2H)} & Reichweite: 2\,m & Max Ang: 1\\
\hline
Angriff: +2 & Schaden: 2Stä+1 & Schleichen: -5\\
\hline
\multicolumn{3}{|p{0.99\textwidth}|}{100\% Rüstungsdurchdringung.} \\
\hline
\end{tabular}
\newline \newline\newline
\begin{tabular}{|p{0.33\textwidth}|p{0.33\textwidth}|p{0.33\textwidth}|}
\hline
\textbf{Kurzbogen (2H)} & Reichweite: 20\,m|35\,m|50\,m & Max Ang: 1 \\
\hline
Angriff: +2 & Schaden: Stä & Schleichen: -2\\
\hline
\multicolumn{3}{|p{0.99\textwidth}|}{Benötigt Max Stä > 5.} \\
\hline
\end{tabular}
\newline \newline\newline
\begin{tabular}{|p{0.33\textwidth}|p{0.33\textwidth}|p{0.33\textwidth}|}
\hline
\textbf{Bogen (2H)} & Reichweite: 25\,m|45\,m|70\,m & Max Ang: 1\\
\hline
Angriff: +1 & Schaden: Stä+1W4 & Schleichen: -3\\
\hline
\multicolumn{3}{|p{0.99\textwidth}|}{Benötigt Max Stä > 7.} \\
\hline
\end{tabular}
\newline \newline\newline
\begin{tabular}{|p{0.33\textwidth}|p{0.33\textwidth}|p{0.33\textwidth}|}
\hline
\textbf{Langbogen (2H)} & Reichweite: 25\,m|50\,m|100\,m & Max Ang: 1\\
\hline
Angriff: +0 & Schaden: Stä+1W6 & Schleichen: -4\\
\hline
\multicolumn{3}{|p{0.99\textwidth}|}{Benötigt Max Stä > 8.} \\
\hline
\end{tabular}
\newline \newline\newline
\begin{tabular}{|p{0.33\textwidth}|p{0.33\textwidth}|p{0.33\textwidth}|}
\hline
\textbf{Kriegsbogen (2H)} & Reichweite: 25\,m|60\,m|120\,m & Max Ang: 1\\
\hline
Angriff: -1 & Schaden: 2Stä & Schleichen: -5\\
\hline
\multicolumn{3}{|p{0.99\textwidth}|}{Benötigt Max Stä > 9.} \\
\hline
\end{tabular}
\newline \newline\newline
\begin{tabular}{|p{0.33\textwidth}|p{0.33\textwidth}|p{0.33\textwidth}|}
\hline
\textbf{Kleine Armbrust (1H)} & Reichweite: 15\,m|25\,m|35\,m & Max Ang: 1 \\
\hline
Angriff: +0 & Schaden: 1W8 & Schleichen: 0\\
\hline
\multicolumn{3}{|p{0.99\textwidth}|}{Nachladen kostet eine Standardaktion.} \\
\hline
\end{tabular}
\newline \newline\newline
\begin{tabular}{|p{0.33\textwidth}|p{0.33\textwidth}|p{0.33\textwidth}|}
\hline
\textbf{Mittlere Armbrust (2H)} & Reichweite: 20\,m|40\,m|60\,m & Max Ang: 1\\
\hline
Angriff: +0 & Schaden: 3W6 & Schleichen: -2\\
\hline
\multicolumn{3}{|p{0.99\textwidth}|}{Nachladen kostet eine volle Aktion.} \\
\hline
\end{tabular}
\newline \newline\newline
\begin{tabular}{|p{0.33\textwidth}|p{0.33\textwidth}|p{0.33\textwidth}|}
\hline
\textbf{Schwere Armbrust (2H)} & Reichweite: 25\,m|50\,m|95\,m & Max Ang: 1\\
\hline
Angriff: +0 & Schaden: 5W6 & Schleichen: -4\\
\hline
\multicolumn{3}{|p{0.99\textwidth}|}{Nachladen kostet zwei volle Aktionen.} \\
\hline
\end{tabular}
\newline \newline\newline
\begin{tabular}{|p{0.33\textwidth}|p{0.33\textwidth}|p{0.33\textwidth}|}
\hline
\textbf{Dolch (1H)} & Reichweite: 0,6\,m & Max Ang: 2\\
\hline
Angriff: +0 & Schaden: Ges+2 & Schleichen: 0\\
\hline
\multicolumn{3}{|p{0.99\textwidth}|}{Dolch kann immer als Schnelle Aktion gezogen werden.} \\
\hline
\end{tabular}
\newline \newline\newline
\begin{tabular}{|p{0.33\textwidth}|p{0.33\textwidth}|p{0.33\textwidth}|}
\hline
\textbf{Schild (1H)} & Reichweite: 0,6\,m & \begin{tabular}{l|l}
Schleichen: -2 & Max Ang: 1
\end{tabular}  \\
\hline
Angriff: +0 & Schaden: Stä & RK: +2\\
\hline
\multicolumn{3}{|p{0.99\textwidth}|}{Schildstoß (Standardaktion): Kann einen Gegner stoßen, bei gewonnener Stemmen/Schieben Probe fällt Ziel zu Boden. 

Während der Schild ausgerüstet ist werden alle Angriffswürfe um 1 reduziert.

Kann im Kampf durch wuchtige Schläge zerstört werden.}\\
\hline
\end{tabular}
\newline \newline\newline
\begin{tabular}{|p{0.33\textwidth}|p{0.33\textwidth}|p{0.33\textwidth}|}
\hline
\textbf{Faustschild (1H)} & Reichweite: 0,6\,m & \begin{tabular}{l|l}
Schleichen: 0 & Max Ang: 1
\end{tabular}  \\
\hline
Angriff: +1 & Schaden: Stä+1 & RK: +1\\
\hline
\multicolumn{3}{|p{0.99\textwidth}|}{} \\
\hline
\end{tabular}

\subsection{Beispielrüstungen}
\begin{multicols}{2}
\begin{tabular}{|p{0.2\textwidth}|p{0.2\textwidth}|}
\hline
\textbf{Gambeson} & RK: +\{2-4\}\\
\hline
\multicolumn{2}{|p{0.4\textwidth}|}{-\{0,2\} Bew, -\{2-5\} Schwimmen} \\
\hline
\end{tabular} 


\begin{tabular}{|p{0.2\textwidth}|p{0.2\textwidth}|}
\hline
\textbf{Kettenhemd} & RK: +\{2-4\}\\
\hline
\multicolumn{2}{|p{0.4\textwidth}|}{-1 Bew, -3 Schwimmen} \\
\hline
\end{tabular}

\begin{tabular}{|p{0.2\textwidth}|p{0.2\textwidth}|}
\hline
\textbf{Lederrüstung} & RK: +2\\
\hline
\multicolumn{2}{|p{0.4\textwidth}|}{-2 Schwimmen, edleres Aussehen} \\
\hline
\end{tabular}
\end{multicols}

\section{Schaden und Regeneration}

\subsection{Fallschaden}
Fällt ein Charakter mindestens 2\,m, muss er eine Akrobatikprobe ablegen. Der Grundfallschaden beträgt
\begin{align*}
	\left(\frac{\text{Höhe}}{\text{m}}-2\right)\text{W}6.
\end{align*}
Ist die Akrobatikprobe erfolgreich wird der Schaden um 1W6 und um je einen weiteren W6 pro Steigerung reduziert.

Dies gilt als grobe Richtlinie, die Umstände können den Fallschaden beeinflussen. Fällt ein Charakter beispielsweise von einem Baum, können Äste seinen Sturz abschwächen.

\subsection{Tauchen/Ersticken}
Wenn Atmung nicht möglich ist, muss ein Charakter bei voller Lunge nach 15*Max Kon sec (3*Max Kon Kampfrunden) alle 5 sec/je Kampfrunde eine Probe auf Ausdauer gegen die Erschwernis von 15 (Esn steigt je Wurf um 2) werfen. Beim ersten Fehlschlag fällt der Charakter in Ohnmacht, je weiteren erhält er eine Narbe auf Bil.

\subsection{Schaden/Ohnmacht/Tod}
Fallen die Lebenspunkte auf 0 oder niedriger, so muss der Charakter eine Probe auf Wille oder Ausdauer gegen die Erschwernis von 9-Lp werfen. Schlägt die Probe fehl, fällt er in Ohnmacht.

Fallen die Lebenspunkte unter -Max Kon, so muss der Charakter eine Probe auf Wille oder Ausdauer gegen die Erschwernis von 10-Lp werfen. Schlägt die Probe fehl, stirbt der Charakter.

In der nächsten Ruhephase wird die Schwere des Schadens bestimmt.
\begin{align*}
 W = \text{PW}\left(\text{Wunden versorgen}\right) + 1\text{W}20
\end{align*}
Der zu erreichende Wert hängt von den verlorenen Lebenspunkten ab. Der Wundenwurf $W$ muss größer als die Zahl in der folgenden Tabelle sein, sonst erhält der Charakter eine Narbe.
\begin{center}
\begin{tabular}{|l|l|}
\hline
$10 \% < \frac{\text{Lp}}{\text{Max Lp}} \le 25 \%$ & $W > 10$ \\ \hline
$0 < \frac{\text{Lp}}{\text{Max Lp}} \le 10 \%$ & $W > 19$\\ \hline
$\frac{\text{Lp}}{\text{Max Lp}} \le 0$ & $W > 28$\\ \hline
\end{tabular}
\end{center}

\subsection{Narben}
Narben verringern den Attributswert um 1. Erleidet ein Charakter an einem Körperteil die dritte Narbe, stirbt das Körperteil ab und der Charakter stirbt ggf. Das betroffene Körperteil wird mit einem W10 ausgewürfelt, außer es ergibt sich aus dem Spielverlauf welches Körperteil die Narbe erhält.

\begin{center}
\begin{tabular}{|l|l|l|}
\hline
Würfelergebnis & Körperteil & Attribut \\ \hline
1 & Augen, Ohren, Nase & War \\ \hline
2 & Hinterkopf & Bil \\ \hline
3 & Gesicht & Cha \\ \hline
4 & Hand & Ges \\ \hline
5 & Becken & Stä \\ \hline
6 & Fuß & Wil \\ \hline
7 & Brustkorb & Kon \\ \hline
8 & Bauch & Hum \\ \hline
9 & Stirn & Int \\ \hline
10 & Bein & Bew \\ \hline
\end{tabular}
\end{center}

\subsection{Heilung}
Nach einem Kampf kann ein Charakter 50\% des erhaltenen Schadens regenerieren indem er verschnauft. Dabei wird pro Minute ein Lebenspunkt regeneriert.

In jeder längeren Ruhephase in dem sich der Charakter ausruht (in der Regel nachts beim Schlafen), kann der Regenerationswert für Lebenspunkte und Mana ausgewürfelt werden und zu den Lebenspunkten bzw. Manapunkten addiert werden.

\end{document}
