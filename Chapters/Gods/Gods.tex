\documentclass[../../Heldenanleitung2]{subfiles}
\begin{document}

\chapter{Götter}
Der vorherrschende Glaube ist der Glaube der Zwölfgötter, ein
Pantheon aus zwölf Geschwistern, dem Praios, der Gott
der Herrschaft und Ordnung, als Götterfürst vorsteht.
Kaum eine Bäuerin begegnet zwar jemals einem Magier
oder einem Ungeheuer, aber sie erlebt häufig die
wundersamen Kräfte ihres Dorfpriesters und sieht, wie
es auf Bitten des Geweihten anfängt zu regnen oder sich
die Wunden eines Verletzten schließen.
Doch selbst die Götter haben Feinde, deren Wirken
sich nur allzu deutlich zeigt: Überall im Land gibt es dunkle Kulte, in denen die Menschen die Erzdämonen anbeten und sie darin
unterstützen, ihren ewigen Kampf gegen die Götter
auszutragen.

Und neben den Erzdämonen ist es vor allem der Gott
ohne Namen, der die Macht der anderen Götter
heimlich unterwandert und seine Diener ausschickt,
um seinen Willen durchzusetzen.
\renewcommand{\arraystretch}{1.5}
{\rowcolors{2}{black!10!white!40}{black!50!white!50}
\begin{table}[h!]
\centering
\begin{tabular}{|l|l|l|}
\hline
\textbf{Gottheit} & \textbf{Aspekte} & \textbf{Heilige Tiere/Gegenstände}\\
\hline
Praios & Gerechtigkeit, Magiebann, Wahrheit & Greif\\
Rondra & Ehre, Kampf, Mut & Löwin\\
Efferd & Stürme, Wasser, Wind & Delphin\\
Travia & Ehe, Gastfreundschaft, Treue & Wildgans\\
Boron & Schlaf, Tod, Vergessen & Rabe \\
Hesinde & Kunsthandwerk, Magie, Wissen & Schlange\\
Firun & Eis, Jagd, Natur & Polarbär\\
Tsa & Jugend, Leben, Wiedergeburt & Eidechse\\
Phex & Diebe, Glück, Handel & Fuchs\\
Peraine & Ackerbau, Heilung, Viehzucht & Storch\\
Ingerimm & Feuer, Handwerk, Schmiedekunst & Hammer und Amboss\\
Rahja & Liebe, Rausch, Schönheit & Pferd\\
\hline
\end{tabular}
\end{table}
}

{\rowcolors{2}{black!10!white!40}{black!50!white!50}
\begin{table}[h!]
\centering
\begin{tabular}{|l|l|l|}
\hline
\textbf{Erzdämon} & \textbf{Aspekte} & \textbf{Zweitname}\\
\hline
Blakharaz & Folter, Dunkelheit, Rache & Tyakra'Man\\
Belhahar & Blutrausch, Heimtücke, Massaker & Xarfai\\
Charyptoroth & Meeresungeheuer, Stürme, verdorbenes Wasser & Gal'k'Zuul\\
Lolgramoth & Rastlosigkeit, Ruhelosigkeit, Treuelosigkeit & Thezzphai\\
Thargunitoth & Alpträume, Nekromantie, Untote & Tijakool\\
Amazeroth & Illusionen, Täuschung, Wahnsinn & Iribaar\\
Nagrach & Gnadenlosigkeit, Kälte, Unbarmherzigkeit & Belshirash\\
Asfaloth & Chaos, Chimären, Verwandlungen & Calijnaar\\
Tasfarelel & Geld, Habgier, Neid & Zholvar\\
Mishkara & Krankheiten, Missernten, Unfruchtbarkeit & Belzhorash\\
Agrimoth & Konstruktionen, Unelemente & Widharcal\\
Belkelel & Orgien, Perversion, Selbstsucht & Dar-Klajid\\
\hline
\end{tabular}
\end{table}
}
\section{Praiosgeweihte}
Die Geweihten des Götterfürsten dienen der Gerechtigkeit und Ordnung.
Sie helfen den Menschen
gegen die Machenschaften
der Diener der Erzdämonen
und gehen gegen
alles Namenlose vor. Prairosgeweihte gehen allgemein gegen den Gebrauch von Magie vor. 

\subsection{Moralkodex der Praiosgeweihten}
\textbf{Gehorsam:} Der Geweihte ist verpflichtet, sich an die
Befehle von Personen zu halten, die über ihm in der
Kirchenhierarchie stehen.\\

\noindent
\textbf{Offensichtlichkeit:} Der Geweihte versteckt sich nicht.\\

\noindent
\textbf{Schutz von Gesetz und Staat:} Der Geweihte verteidigt
zwölfgöttliche Reiche und Strukturen und achtet auf
die Einhaltung der Gesetze.\\

\noindent
\textbf{Magiebann:} Magie, vor allem schwarze Magie (Magie die den Geist verändert, Schaden anrichtet oder die Toten schändet), sollte gebannt
werden. Weiße Magie kann unter Umständen toleriert
werden.\\

\noindent
\textbf{Mission:} Der Glaube an den Götterfürsten und seine Geschwister
muss in alle Winkel verbreitet werden.

\subsection{Praiosgeweihter als Talent}
\begin{itemize}
	\item[Rang 1] Ein Praiosgeweihter muss sich an den Moralkodex
(Prinzipientreue) halten, sonst verliert er alle Vorteile des Talents.
	\item[Rang 1] Alle Proben die geworfen werden um einem Zauber zu widerstehen erhalten einen Bonus von 2.
	\item[Rang 1] +1 Überzeugen, Willenskraft, Magiekunde
	\item[Rang 2] Jeglicher magischer Schaden wird halbiert.
	\item[Rang 2] +1 Einschüchtern, Menschenkenntnis, Orientierung
\end{itemize}

\section{Rondrageweihte}
Der Kirche der Rondra
steht das Schwert
der Schwerter vor. Die
oberste Geweihte leitet
die Geschicke der
durch Krieg und Kämpfe
geschrumpften Glaubensgemeinschaft.
Vor
allem der Adel ehrt Rondra
noch immer als seine
Schutzpatronin.
Für die Geweihten hat der Schutz von Gläubigen und
Tempeln oberste Priorität und nicht selten geben sie ihr
Leben im Kampf gegen unheilige Mächte.
Ebenfalls in ihrem Interesse sind der ewige Kampf
und die Verbesserung der Schwertkunst und anderer
Kampftechniken. Nur ein wehrhafter Rondrianer ist in
den Augen der Göttin würdig, eines Tages in ihr Paradies
einzuziehen.

\subsection{Moralkodex der Rondrageweihten}
\textbf{Verteidigung des Glaubens:} Die Verteidigung des
Glaubens ist die Pflicht jedes Rondrageweihten.\\

\noindent
\textbf{Ritterlichkeit:} Der Rondrageweihte setzt im Kampf
keine Armbrüste ein oder verhält sich unehrenhaft.\\

\noindent
\textbf{Verantwortung:} Der Schutz aller Gläubigen, der
Heiligtümer und der Tempel der Zwölfgötter steht im
Vordergrund der Aufgaben eines Geweihten.\\

\noindent
\textbf{Zweikampf:} Der ehrenhafte Zweikampf ist von allen
Kampfhandlungen die ehrenhafteste.\\

\noindent
\textbf{Schwertmeisterschaft:} Sich in allen Waffengattungen
auszukennen und sie zu meistern ist eine Selbstverständlichkeit
für den Geweihten.

\subsection{Rondrageweihter als Talent}
\begin{itemize}
	\item[Rang 1] Ein Rondrageweihter muss sich an den Moralkodex
(Prinzipientreue) halten.
	\item[Rang 1] Das Verängstigtlevel wird immer um 2 reduziert und Furcht wird wie verängstigt Level 3 behandelt.
	\item[Rang 1] +1 auf 3 beliebige Nahkampfwaffen
	\item[Rang 2] Immunität gegen Verängstigt und Furcht.
	\item[Rang 2] +1 Überwältigen, Kulturwissen, Taktisches Wissen
\end{itemize}

\section{Borongeweihte}
Die Geweihten des Boron
sind für die Menschen
da, wenn es darum geht,
einen Verwandten oder
Freund zu bestatten und
ihm die letzte Ehre zu erweisen.
Sie unterhalten
die Boronanger, die Bezeichnung für
einen Friedhof, wo sie über
die Totenruhe wachen.
Ihre ärgsten Feinde sind Geisterbeschwörer
und Nekromanten. Die Kirche des Boron ist vor
langer Zeit in zwei Teile gespalten worden. Der al'anfanische
Teil verehrt Boron als den obersten Gott
der Zwölfe, während der punische Teil des Kultes eine gemäßigtere
Einstellung besitzt.
Nicht selten reist ein Borongeweihter durch die Lande,
um Tote zu begraben und unheilvolle Nekromantie
zu bekämpfen. Dabei werden die Geweihten oftmals
von dem Orden der Golgariten unterstützt, einem
borongefälligen Bund von Ordenskriegern in schwarzen
Plattenrüstungen.

\subsection{Moralkodex der Borongeweihten}
\textbf{Bestattung:} Kein Leichnam sollte unbestattet sein. Der
Geweihte muss für die Totenruhe sorgen.\\

\noindent
\textbf{Schweigen:} Schweigen ist eine Tugend. Kein Geweihter
sollte ohne Grund reden.\\

\noindent
\textbf{Traum:} Ergründe die Welt der Träume. In ihr kann der
Geweihte den Willen Borons ergründen.

\subsection{Borongeweihter als Talent}
\begin{itemize}
	\item[Rang 1] Ein Boronweihter muss sich an den Moralkodex
(Prinzipientreue) halten.
	\item[Rang 1] Waffen verursachen doppelten Schaden an Untoten.
	\item[Rang 1] +1 Schleichen, Einschüchtern, Willenskraft, Medizin, Kulturkunde
	\item[Rang 2] +1 auf den Flankierenbonus
	\item[Rang 2] +1 Beruhigen, +1 Geografie, +1 Naturmaterialien
\end{itemize}

\section{Hesindegeweihte}
Das Sammeln und Bewahren
von Wissen ist für die
Hesindegeweihten eine
heilige Pflicht. Sie kümmern sich um Archive und
unternehmen weite Reisen,
um längst vergessene Mysterien
zu ergründen und gefährliche magische Artefakte zu
bergen.

In der Stadt Kuslik steht die größte Bibliothek der
Kirche, und hier lebt auch die oberste Geweihte, die
Magisterin der Magister, und leitet ihre Kirche mit
ausgleichender Weisheit. Innerhalb der Kirche gibt
es nämlich zwei große Strömungen: Die eine will das
erbeutete Wissen wegsperren, die andere es verbreiten,
um es für Göttergefälliges einzusetzen.

\subsection{Moralkodex der Hesindegeweihten}
\textbf{Sammeln von Wissen:} Artefakte, Bücher und anderes
Wissen sind in den Augen der Göttin so wertvoll, dass
es gesammelt werden muss.\\

\noindent
\textbf{Ewige Lehre:} Der Geweihte sollte sich bemühen, sich
stets fortzubilden.\\

\noindent
\textbf{Ästhetik:} Die Welt ist schön und der Geweihte soll die
Schönheit der Welt ehren und mehren.

\subsection{Hesindegeweihter als Talent}
\begin{itemize}
	\item[Rang 1] Ein Hesindegeweihter muss sich an den Moralkodex
(Prinzipientreue) halten.
	\item[Rang 1] Verzauberungen die sich gegen einen Hesindegeweihten richten erhalten einen Malus von 2 auf den Zauberprobenwurf
	\item[Rang 1] +1 Bemerken, 1 beliebige Wissenprobe, Überzeugen
	\item[Rang 2] +1 Magie spüren, 2 beliebige Wissenproben, Alchemie, Willenskraft
\end{itemize}

\section{Phexgeweihte}

Der Gott der Diebe
verfügt über eine sehr
individuelle Geweihtenschaft.
Viele Phexgeweihte
sind selbst
so etwas wie Diebe und
Fassadenkletterer, die
ihrem Gott die wertvollsten
Schätze als Opfer
darbringen wollen und
jede Nacht ihr Können erneut
auf die Probe stellen.
Andere hingegen sind mehr Händler und kümmern sich
um die Belange anderer Händler, schreiben Vertragsabschlüsse
oder organisieren Karawanen und Handelszüge.
Phexgeweihte machen jedoch nichts ohne eine Gegenleistung.
Der oberste Geweihte der Kirche wird der Mond
genannt. Wer sich jedoch dahinter verbirgt, ist selbst
den Geweihten des Phex unbekannt.

\subsection{Moralkodex der Phexgeweihten}
\textbf{Gegenleistung:} Für eine zu erfüllende Aufgabe muss
der Geweihte stets eine Gegenleistung verlangen.\\

\noindent
\textbf{Heimlichkeit:} Der Geweihte soll seine Pläne im Verborgenen
ausführen.\\

\noindent
\textbf{Herausforderung:} Je größer die Herausforderung, desto
größer der Ruhm. Der Geweihte soll große Herausforderungen
suchen und sich ihnen stellen.

\subsection{Phexgeweihter als Talent}
\begin{itemize}
	\item[Rang 1] Ein Phexgeweihter muss sich an den Moralkodex
(Prinzipientreue) halten.
	\item[Rang 1] Ein mal je Session kann ein Würfelwurf wiederholt werden und das bessere Ergebnis verwendet werden.
	\item[Rang 1] +1 Schleichen, Taschendiebstahl, Kletter
	\item[Rang 2] +1 Schlossknacken, Überzeugen/Verhandeln, Gossenwissen
	\item[Rang 2] Der Phexgeweihte kann 10 Silber an seinen Gott opfern um sich und seine KLeidung in Rauch zu verwandeln. Bei Windstille kann er sich mit normaler Geschwindigkeit fortbewegen. Bei leichtem Wind ist eine Willenskraftprobe (SK 9) erforderlich, um nicht davon geweht zu werden. Bei starkem Wind wird der Geweihte weggeweht.
\end{itemize}

\section{Perainegeweihte}

Die als friedliebend bekannten
Geweihten der
Peraine sind bekannt
für ihre Heilkünste.
Doch sie sind nicht nur
bereit, Bedürftigen zu
helfen, sondern auch
profunde Kenner von
Giften und Krankheiten.
Schon so mancher Geweihte
konnte den Ausbruch einer
Seuche bekämpfen und so ganze Dörfer vor der Ausrottung
retten. Den Geweihten wird zudem ein gesunder
Pragmatismus nachgesagt, und sie sind auch bereit,
über die Grenzen ihres Glaubens hinaus zu helfen und
Nichtzwölfgöttergläubige versorgen sie ebenfalls. Außerdem sind sie tatkräftige Helfer der 
Bauern und helfen bei der Aussaat und der Ernte von
Korn, Heilpflanzen und anderen Erzeugnissen.

\subsection{Moralkodex der Perainegeweihten}
\textbf{Hilfe:} Leiste jedem Hilfe, der Hilfe benötigt.\\

\noindent
\textbf{Aufopferung:} Arbeite hart und halte dich vom
Müßiggang fern.\\

\noindent
\textbf{Bescheidenheit:} Verschwende nicht die Gaben der
Göttin.\\

\noindent
\textbf{Heilkunst:} Bilde dich in der Heilkunst fort.

\subsection{Perainegeweihter als Talent}
\begin{itemize}
	\item[Rang 1] Ein Perainegeweihter muss sich an den Moralkodex
(Prinzipientreue) halten.
	\item[Rang 1] Ein mal pro Tag Session kann ein Individuum gesegnet werden. Dieses stellt 1W6 pro Rang LP über 6 Stunden wieder her.
	\item[Rang 1] +1 Biologie, Medizin, Menschenkenntnis
	\item[Rang 2] +1 Willenskraft, Überzeugen/Verhandeln, Kulturkunde
	\item[Rang 2] Der Perainegeweihte ist immun gegen Krankheiten und Gifte sind gegen ihn nur halb so wirksam.
\end{itemize}
\end{document}