\documentclass[../../Heldenanleitung2]{subfiles}
\begin{document}

\chapter{Magie}

Alle Zauber sind in die Magieschulen Verzauberung, Beschwörung, Zerstörung und Veränderung, eingeteilt. Wenn ein Zauber gewirkt wird, muss eine Zauberprobe der jeweiligen Magieschule geworfen werden. Jeder Zauber hat entweder eine individuelle Schwierigkeit (SK) oder es ist eine Gegenprobe (GP) vom Ziel durchzuführen um ihn abzuwenden. Einen Zauber durchzuführen kostet immer Mana. Zauber bei denen Manakosten pro Runde angegeben sind, müssen aktiv aufrechtgehalten werden. Zauber können gesteigert werden. Für je 4 Punkte die der Probenwert (PW) der Zauberprobe über der SK ist, kann der Zauber gesteigert werden. Ist zum Beispiel die SK 6 und der PW 17, sind 2 Steigerungen erreicht. Für Fernangriffszauber, bei denen 3 Reichweiten angegeben sind und gegen die RK bzw. AU des Ziels gewürfelt wird, gelten die üblichen Regeln für Fernangriffe.
\\

\renewcommand{\arraystretch}{1.2}
\begin{tabular}{|c|c|c|c|}
\hline
\textbf{Verzauberung} & \textbf{Beschwörung} & \textbf{Zerstörung} & \textbf{Veränderung} \\ \hline
Panik & Lichtkugel & Flammenhand & Pflanzenschub\\
Tiere besänftigen & Blenden & Wirbelsturm & Pfütze vertiefen\\
Dunkelsicht & Schildzauber & Energiegeschoss & Gefrieren\\
Langsamer Fall & Wand &  & Alarm\\
Tiersprache & Rankenangriff  &  & Elementare Manipulation\\ 
Totensprache & Sumpf &  & Tonmanipulation\\
\hline
Verwirrung & Schutzgeist  & Feuerball  & Telekinese \\
Blut kochen  & Wurzelfessel  & Energieschwert  & Dunkelheit \\
Sicht verhüllen & Schutzfeld  & Kettenblitz  & Eisgegenstand \\
Einschläfern & Spiegelbild  & Eisgeschoss  & Extreme Temperatur \\
Weitsicht & Verkleidung   & Desintegration & Inhalt tauschen\\
Befreunden & Arkane Hände  & Erdbeben & Vergraben\\
Unsichtbarkeit & Nebel  &  & Regeneration\\
Krankheit & Tote erwecken &  & Magie bannen\\
 & Giftwolke &  & Reinigung\\
 &  &  & Stille\\\hline
Fluch  &  Elementar beschwören & Flammenmeer  & Antimagiefeld\\
Blutrausch  & Totendiener &  & Bärengestalt\\
Blutritual  &  &  & \\\hline
 
\end{tabular}

\section{Verzauberung}
Die Schule der Verzauberung beinhaltet Zauber, die sich auf Lebewesen beziehen und sie beeinflussen.
\\\\\\
\begin{tabular}{|p{\textwidth}|}
\hline
\begin{tabularx}{\textwidth}{X|X|X|X}
\textbf{Panik} & Rang 1 & 2 Mana/Ziel & SK: Willenskraft - 2
\end{tabularx} \\ \hline
\begin{tabularx}{\textwidth}{X|X|X}
Volle Aktion & Reichweite: 15 & Dauer: bis Befreiung
\end{tabularx} \\ \hline
ZW löst den Zustand Panik in bis zu 3 Lebewesen in bis zu 15 m Reichweite aus. Panische Lebewesen ergreifen sofort die Flucht. Betroffene werfen zu Beginn jeder ihrer Runden auf Willenskraft gegen den PW+2 des Zaubers. Nach jeder gescheiterten Probe wird sie um 2 leichter.
\\ \hline
\end{tabular}
\\\\\\
\begin{tabular}{|p{\textwidth}|}
\hline
\begin{tabularx}{\textwidth}{X|X|X|X}
\textbf{Tiere besänftigen} & Rang 1 & 2 Mana & SK: SL
\end{tabularx} \\ \hline
\begin{tabularx}{\textwidth}{X|X|X}
Standardaktion & Reichweite: 20 & Dauer: 1 Stunde
\end{tabularx} \\ \hline
ZW kann ein Tier in 20\,m Entfernung beruhigen, dieses ist nicht mehr aggressiv.
\\ \hline
\end{tabular}
\\\\\\
\begin{tabular}{|p{\textwidth}|}
\hline
\begin{tabularx}{\textwidth}{X|X|X|X}
\textbf{Dunkelsicht} & Rang 1 & 1 Mana/Stunde & SK: 8
\end{tabularx} \\ \hline
\begin{tabularx}{\textwidth}{X|X|X}
Standardaktion & Reichweite: selbst & Dauer: Manakosten h
\end{tabularx} \\ \hline
ZW kann im Dunkeln sehen. Alle Mali durch Dunkelheit werden um 4(+1xSteigerung) verringert.
\\ \hline
\end{tabular}
\\\\\\
\begin{tabular}{|p{\textwidth}|}
\hline
\begin{tabularx}{\textwidth}{X|X|X|X}
\textbf{Langsamer Fall} & Rang 1 & 2 Mana & SK: 8
\end{tabularx} \\ \hline
\begin{tabularx}{\textwidth}{X|X|X}
Standardaktion & Reichweite: selbst & Dauer: 1\,min
\end{tabularx} \\ \hline
Langsamer Fall reduziert die Fallgeschwindigkeit auf maximal 2\,m/s. Beim Aufprall wird kein Schaden erlitten.
\\ \hline
\end{tabular}
\\\\\\
\begin{tabular}{|p{\textwidth}|}
\hline
\begin{tabularx}{\textwidth}{X|X|X|X}
\textbf{Tiersprache} & Rang 1 & 1 Mana & SK: 7/9
\end{tabularx} \\ \hline
\begin{tabularx}{\textwidth}{X|X|X}
Standardaktion & Reichweite: 10\,m & Dauer: 30\,min
\end{tabularx} \\ \hline
ZW und alle Verbündeten in 10\,m Entfernung können mit einer bestimmten Tierartsprechen (SK: 7) oder mit allen Tieren (SK: 9).
\\ \hline
\end{tabular}
\\\\\\
\begin{tabular}{|p{\textwidth}|}
\hline
\begin{tabularx}{\textwidth}{X|X|X|X}
\textbf{Totensprache} & Rang 1 & 1 Mana & SK: 8
\end{tabularx} \\ \hline
\begin{tabularx}{\textwidth}{X|X|X}
Standardaktion & Reichweite: 10\,m & Dauer: 30\,min
\end{tabularx} \\ \hline
ZW kann mit einer Leiche sprechen die noch immer einen Mund hat und nicht älter als ein halbes Jahr ist und auf die dieser Zauber nicht in den letzten 10 Tagen angewandt wurde. Die Leiche weiß nur was sie im Leben wusste und kann 5 Fragen beantworten. Die Antworten sind in der Regeln nur sehr kurz und oft etwas kryptisch.
\\ \hline
\end{tabular}
\\\\\\
\begin{tabular}{|p{\textwidth}|}
\hline
\begin{tabularx}{\textwidth}{X|X|X|X}
\textbf{Verwirrung} & Rang 2 & 3 Mana & SK: Willenskraft-2
\end{tabularx} \\ \hline
\begin{tabularx}{\textwidth}{X|X|X}
Volle Aktion & Reichweite: 20 & Dauer: bis Befreiung
\end{tabularx} \\ \hline
ZW löst eine Verwirrung in einem Lebewesen in 20 m Reichweite aus. Das Opfer wirft zu Beginn jeder seiner Runden auf Willenskraft gegen den PW+2 des Zaubers. Gelingt dies nicht greifen sie zufällige Lebewesen in der Umgebung an. Nach jeder gescheiterten Probe wird sie um 2 leichter.
\\ \hline
\end{tabular}
\\\\\\
\begin{tabular}{|p{\textwidth}|}
\hline
\begin{tabularx}{\textwidth}{X|X|X|X}
\textbf{Blut kochen} & Rang 2 & 2 Mana pro Runde & SK: Willenskraft
\end{tabularx} \\ \hline
\begin{tabularx}{\textwidth}{X|X|X}
Volle Aktion & Reichweite: 20 & Dauer: beliebig
\end{tabularx} \\ \hline
ZW lässt das Blut eines Ziels kochen. Richtet immer wenn das Ziel an der Reihe ist Wil(+1xSteigerung) Magieschaden an.
\\ \hline
\end{tabular}
\\\\\\
\begin{tabular}{|p{\textwidth}|}
\hline
\begin{tabularx}{\textwidth}{X|X|X|X}
\textbf{Sicht verhüllen} & Rang 2 & 2 Mana & SK: GP Reagieren
\end{tabularx} \\ \hline
\begin{tabularx}{\textwidth}{X|X|X}
Standardaktion & Reichweite: 30 & Dauer: 5 Runden
\end{tabularx} \\ \hline
ZW legt einen Schatten über das Ziel. Das Ziel hat eine schlechte Sicht (Bei einer Steigerung: ist fast blind, bei zwei Steigerungen ist blind). Das Ziel bekommt -2(-1xSteigerung) auf die RK. Ab einer Steigerung kann das Ziel keine Fernkampfangriffe mehr durchführen.
\\ \hline
\end{tabular}
\\\\\\
\begin{tabular}{|p{\textwidth}|}
\hline
\begin{tabularx}{\textwidth}{X|X|X|X}
\textbf{Einschläfern} & Rang 2 & 2 Mana & SK: Willenskraft
\end{tabularx} \\ \hline
\begin{tabularx}{\textwidth}{X|X|X}
Volle Aktion & Reichweite: 20 & Dauer: bis Ziel geweckt wird
\end{tabularx} \\ \hline
ZW versetzt ein Lebewesen in natürlichen Schlaf. 
\\ \hline
\end{tabular}
\\\\\\
\begin{tabular}{|p{\textwidth}|}
\hline
\begin{tabularx}{\textwidth}{X|X|X|X}
\textbf{Wasseratmung} & Rang 2 & 3 Mana & SK: 9
\end{tabularx} \\ \hline
\begin{tabularx}{\textwidth}{X|X|X}
Standardaktion & Reichweite: selbst & Dauer: 20 min + 2 min/Lvl
\end{tabularx} \\ \hline
ZW kann unter Wasser atmen.. 
\\ \hline
\end{tabular}
\\\\\\
\begin{tabular}{|p{\textwidth}|}
\hline
\begin{tabularx}{\textwidth}{X|X|X|X}
\textbf{Weitsicht} & Rang 2 & 3 Mana & SK: 9
\end{tabularx} \\ \hline
\begin{tabularx}{\textwidth}{X|X|X}
Standardaktion & Reichweite: selbst & Dauer: 10 min + 1 min/Lvl
\end{tabularx} \\ \hline
ZW kann Details in doppelter Entfernung erkennen. Distanzmali werden halbiert.
\\ \hline
\end{tabular}
\\\\\\
\begin{tabular}{|p{\textwidth}|}
\hline
\begin{tabularx}{\textwidth}{X|X|X|X}
\textbf{Befreunden} & Rang 2 & 3 Mana & SK: Willenskraft
\end{tabularx} \\ \hline
\begin{tabularx}{\textwidth}{X|X|X}
Standardaktion & Reichweite: 5 & Dauer: 5\,min (+2xSteigerung)
\end{tabularx} \\ \hline
Ein Lebewesen ist für 5\,min(+2xSteigerung) der beste Freund des ZW.
\\ \hline
\end{tabular}
\\\\\\
\begin{tabular}{|p{\textwidth}|}
\hline
\begin{tabularx}{\textwidth}{X|X|X|X}
\textbf{Unsichtbarkeit} & Rang 2 & 5 Mana & SK: 10
\end{tabularx} \\ \hline
\begin{tabularx}{\textwidth}{X|X|X}
Volle Aktion & Reichweite: selbst & Dauer: 10\,min
\end{tabularx} \\ \hline
Der Zauberwirker kann sich selbst unsichtbar machen.
\\ \hline
\end{tabular}
\\\\\\
\begin{tabular}{|p{\textwidth}|}
\hline
\begin{tabularx}{\textwidth}{X|X|X|X}
\textbf{Krankheit} & Rang 2 & 4 Mana & SK: Ausdauer
\end{tabularx} \\ \hline
\begin{tabularx}{\textwidth}{X|X|X}
Volle Aktion & Reichweite: 10 & Dauer: 1 Woche
\end{tabularx} \\ \hline
Ein Ziel in 10\,m Entfernung wirft eine Ausdauerprobe gegen den PW des Zaubers. Bei Misserfolg erkrankt das Ziel für eine Woche. Das Ziel erhält einen Malus von 1W4 auf Kon und verliert 3W6 maximale Lebenspunkte.
\\ \hline
\end{tabular}
\\\\\\
\begin{tabular}{|p{\textwidth}|}
\hline
\begin{tabularx}{\textwidth}{X|X|X|X}
\textbf{Blutrausch} & Rang 3 & 2 Mana & SK: 9
\end{tabularx} \\ \hline
\begin{tabularx}{\textwidth}{X|X|X}
Bonusaktion & Reichweite: Berühren & Dauer: 2 Runden + 1 Runde/Lvl
\end{tabularx} \\ \hline
Versetzt ein Lebewesen in Berührenreichweite in Blutrausch, sodass dieses 4(+4xSteigerung) mehr Schaden verursacht und erhält. Dieser Effekt wird um den gleichen Wert erhöht sobald der vom Blutrausch betroffene einen Gegner tötet. Der Zauber dauert bis zu 2 Runden + 1 Runde pro Spielerlevel an, er kann aber auch früher vom Zauberwirker beendet werden.
\\ \hline
\end{tabular}
\\\\\\
\begin{tabular}{|p{\textwidth}|}
\hline
\begin{tabularx}{\textwidth}{X|X|X|X}
\textbf{Fluch} & Rang 3 & 5 Mana & SK: Willenskraft - 2
\end{tabularx} \\ \hline
\begin{tabularx}{\textwidth}{X|X|X}
Volle Aktion & Reichweite: - & Dauer: 1 Tag
\end{tabularx} \\ \hline
ZW braucht etwas vom Opfer (z.B. Haar). Alle Proben des Opfers sind für einen Tag um 4(+4xSteigerung) erschwert und das Opfer erhält einen Malus von 3(+3xSteigerung) auf alle Angriffswürfe.
\\ \hline
\end{tabular}
\\\\\\
\begin{tabular}{|p{\textwidth}|}
\hline
\begin{tabularx}{\textwidth}{X|X|X|X}
\textbf{Blutritual} & Rang 3 & 4 Mana & SK: 10
\end{tabularx} \\ \hline
\begin{tabularx}{\textwidth}{X|X|X}
2 Volle Aktionen & Reichweite: selbst & Dauer: 3 Tage
\end{tabularx} \\ \hline
ZW kann mit Blut eines frisch getöteten Lebewesens ein Ritual durchführen (mind 1\,l Blut). ZW erhält +2(+2xSteigerung) auf ein beliebiges Attribut für 3 Tage. Bei erneuter Durchführung des Rituals geht der alte Bonus verloren.
\\ \hline
\end{tabular}

\section{Beschwörung}

Die Schule der Beschwörung beschäftigt sich mit dem Erzeugen magischer Kräfte aus dem nichts.
\\\\\\
\begin{tabular}{|p{\textwidth}|}
\hline
\begin{tabularx}{\textwidth}{X|X|X|X}
\textbf{Lichtkugel} & Rang: 1 & 1 Mana & SK: 6
\end{tabularx} \\ \hline
\begin{tabularx}{\textwidth}{X|X|X}
Bonusaktion & Reichweite: 10 & Dauer: 10\,min
\end{tabularx} \\ \hline
ZW beschwört Lichtkugel für 10 min, die der ZW frei im Radius von 10\,m bewegen kann.
\\ \hline
\end{tabular}
\\\\\\
\begin{tabular}{|p{\textwidth}|}
\hline
\begin{tabularx}{\textwidth}{X|X|X|X}
\textbf{Blenden} & Rang 1 & 2/3/4 Mana & SK: GP Reagieren
\end{tabularx} \\ \hline
\begin{tabularx}{\textwidth}{X|X|X}
Bonusaktion & Reichweite: 10/20/30 & Dauer: Sofort
\end{tabularx} \\ \hline
Alle Lebewesen im Kegel (Winkel $120^\circ$) vor dem ZW werden geblendet, sofern sie in Richtung des ZW schauen, und erhalten -1(-1xSteigerung) auf RK und Angriffswürfe für 2 Kampfrunden. Die Standardreichweite von 10\,m kostet den Zauberwirker 2 Mana. Jede höhere Reichweitenklasse erhöht die Kosten um 1 Mana.
\\ \hline
\end{tabular}
\\\\\\
\begin{tabular}{|p{\textwidth}|}
\hline
\begin{tabularx}{\textwidth}{X|X|X|X}
\textbf{Schildzauber} & Rang 1 & 3 Mana & SK: 6
\end{tabularx} \\ \hline
\begin{tabularx}{\textwidth}{X|X|X}
Standardaktion & Reichweite: Berühren & Dauer: 1 Runde/Lvl
\end{tabularx} \\ \hline
Wirkbar auf ein Ziel in Berührenreichweite. Ziel erhält +1(+1xSteigerung) auf RK für 1 Runde pro Spielerlevel.
\\ \hline
\end{tabular}
\\\\\\
\begin{tabular}{|p{\textwidth}|}
\hline
\begin{tabularx}{\textwidth}{X|X|X|X}
\textbf{Rankenangriff} & Rang 1 & 2 Mana & SK: 5
\end{tabularx} \\ \hline
\begin{tabularx}{\textwidth}{X|X|X}
Standardaktion & Reichweite: 30 & Dauer: dauerhaft
\end{tabularx} \\ \hline
ZW beschwört Ranke aus dem Boden in max 30 m Entfernung. Ranke greift alle Feinde in 2\,m Radius, direkt nach dem Zug des Zaubernden, an (zählt als Nahkampfangriff). Als Angriffswurf wird erneut eine Zaubernprobe geworfen. Die Ranke verursacht 2W4(+1xSteigerung) Schaden. Die Ranke hat 10 Lp und keine Abwehr.
\\ \hline
\end{tabular}
\\\\\\
\begin{tabular}{|p{\textwidth}|}
\hline
\begin{tabularx}{\textwidth}{X|X|X|X}
\textbf{Wand} & Rang 1 & 1 Mana/m & SK: 9
\end{tabularx} \\ \hline
\begin{tabularx}{\textwidth}{X|X|X}
Standardaktion & Reichweite: 5 & Dauer: 20 min + 2 min/Lvl
\end{tabularx} \\ \hline
ZW erzeugt eine durchsichtige solide Wand. Die Wand ist bis zu 2\,m hoch und 1\,m breit für jeden Manapunkt der bezahlt wurde. ZW kann jederzeit den Zauber aufheben und die Wand auflösen.
\\ \hline
\end{tabular}
\\\\\\
\begin{tabular}{|p{\textwidth}|}
\hline
\begin{tabularx}{\textwidth}{X|X|X|X}
\textbf{Sumpf} & Rang 1 & 3 Mana & SK: 8
\end{tabularx} \\ \hline
\begin{tabularx}{\textwidth}{X|X|X}
Standard Aktion & Reichweite: 30 & Dauer: 10\,min + 2\,min/Lvl
\end{tabularx} \\ \hline
ZW erschafft einen Sumpf in bis zu 30\,m Entfernung mit einem Radius von bis zu 5\,m. Alle Individuen in dem Sumpf sind verlangsamt. Bei 2 Steigerungen sind alle Individuen stattdessen stark verlangsamt.
\\ \hline
\end{tabular}
\\\\\\
\begin{tabular}{|p{\textwidth}|}
\hline
\begin{tabularx}{\textwidth}{X|X|X|X}
\textbf{Nebel} & Rang 2 & 2-6 Mana & SK: 8
\end{tabularx} \\ \hline
\begin{tabularx}{\textwidth}{X|X|X}
Standard Aktion & Reichweite: 10/Mana & Dauer: 10\,min + 2\,min/Lvl
\end{tabularx} \\ \hline
ZW erschafft einen Nebel in 10\,m Umkreis pro bezahltem Mana. Der Nebel lässt eine Blickweite von 5\,m (-1\,m x Steigerung) zu
\\ \hline
\end{tabular}
\\\\\\
\begin{tabular}{|p{\textwidth}|}
\hline
\begin{tabularx}{\textwidth}{X|X|X|X}
\textbf{Schutzgeist} & Rang 2 & 7 Mana & SK: 9
\end{tabularx} \\ \hline
\begin{tabularx}{\textwidth}{X|X|X}
Volle Aktion & Reichweite: 10\,km & Dauer: 30 min
\end{tabularx} \\ \hline
ZW beschwört einen Wolfsgeist für 30 min, der dem ZW gehorcht und gedanklich Befehle entgegen nimmt. Der Geist kann Nahkampfangriffe ausführen. Für einen Angriff wirft der ZW eine Zaubernprobe. Der Geist verursacht 2W8 Schaden. Der Geist besitzt 20 (+20xSteigerung) Hp und RK 0. Bewegungsweite 6\,m.
\\ \hline
\end{tabular}
\\\\\\
\begin{tabular}{|p{\textwidth}|}
\hline
\begin{tabularx}{\textwidth}{X|X|X|X}
\textbf{Wurzelfessel} & Rang 2 & 3 Mana & SK: GP Stemmen
\end{tabularx} \\ \hline
\begin{tabularx}{\textwidth}{X|X|X}
Standardaktion & Reichweite: 20 & Dauer: dauerhaft
\end{tabularx} \\ \hline
ZW lässt bis zu 2 Wurzeln aus dem Boden schießen um Objekte/Körperteile in Bodennähe zu fixieren. Um sich aus der Fessel zu befreien ist eine Probe auf Stemmen/Schieben gegen die Zauberprobe erforderlich.
\\ \hline
\end{tabular}
\\\\\\
\begin{tabular}{|p{\textwidth}|}
\hline
\begin{tabularx}{\textwidth}{X|X|X|X}
\textbf{Schutzfeld} & Rang 2 & 2 Mana & SK: 8
\end{tabularx} \\ \hline
\begin{tabularx}{\textwidth}{X|X|X}
Standardaktion & Reichweite: 2 & Dauer: 1 Runde/Lvl
\end{tabularx} \\ \hline
ZW erzeugt eine Schützende Sphäre mit 2\,m Radius, die jedes Geschoss, welches die Sphäre durchdringt um 2(+2xSteigerung) Schaden abschwächt.
\\ \hline
\end{tabular}
\\\\\\
\begin{tabular}{|p{\textwidth}|}
\hline
\begin{tabularx}{\textwidth}{X|X|X|X}
\textbf{Spiegelbild} & Rang 2 & 4 Mana & SK: 8
\end{tabularx} \\ \hline
\begin{tabularx}{\textwidth}{X|X|X}
Volle Aktion & Reichweite: selbst & Dauer: 30 min
\end{tabularx} \\ \hline
ZW erzeugt ein Spiegelbild von sich selbst, so dass nicht zu erkennen ist, wer der Echte ist. Das Spiegelbild kann sich bewegen, ist aber stumm und kann keine Zauber ausführen oder Schaden verursachen. Das Spiegelbild kann sich nicht weiter als 20(+10xSteigerung)\,m vom ZW entfernen.
\\ \hline
\end{tabular}
\\\\\\
\begin{tabular}{|p{\textwidth}|}
\hline
\begin{tabularx}{\textwidth}{X|X|X|X}
\textbf{Verkleidung} & Rang 2 & 3-5 Mana & SK: 9
\end{tabularx} \\ \hline
\begin{tabularx}{\textwidth}{X|X|X}
Standardaktion & Reichweite: selbst & Dauer: 30 min + 5 min/Lvl
\end{tabularx} \\ \hline
ZW nimmt die Gestalt einer anderen Person an. Die Basiskosten sind 3 Mana. Ist die zu imitierende Person größer, steigen die Kosten (SL). Die zu imitierende Person kann nicht mehr als doppelt so groß sein wie der ZW. Die Verkleidung kann mit einer Bemerkenprobe durchschaut werden, dazu muss die Bemerkenprobe die Zauberprobe um 2 übertreffen, wenn die imitierte Person gut bekannt ist sonst muss sie um 4 übertroffen werden.
\\ \hline
\end{tabular}
\\\\\\
\begin{tabular}{|p{\textwidth}|}
\hline
\begin{tabularx}{\textwidth}{X|X|X|X}
\textbf{Arkane Hände} & Rang 2 & 2 Mana & SK: 8
\end{tabularx} \\ \hline
\begin{tabularx}{\textwidth}{X|X|X}
Standardaktion & Reichweite: 15 & Dauer: PW min
\end{tabularx} \\ \hline
ZW beschwört ein Paar Hände in maximal 15\,m Entfernung. Die Hände können vom ZW wie die eigenen benutzt werden. Die Hände können innerhalb der 15\,m Reichweite frei bewegt werden
\\ \hline
\end{tabular}
\\\\\\
\begin{tabular}{|p{\textwidth}|}
\hline
\begin{tabularx}{\textwidth}{X|X|X|X}
\textbf{Tote Erwecken} & Rang 2 & 5 Mana & SK: 9
\end{tabularx} \\ \hline
\begin{tabularx}{\textwidth}{X|X|X}
1 min & Reichweite: 5 & Dauer: 24 Stunden
\end{tabularx} \\ \hline
Um den Zauber durchführen zu können sind entweder Knochen oder eine Leiche erforderlich. Werden Knochen verwendet entsteht ein Skelett und mit einer Leiche entsteht ein Zombie. Der Untote kann einfache Befehle telepathisch in 60\,m Entfernung als Bonusaktion empfangen. Der Untote ist nicht in der Lage mit anderen Lebewesen zu kommunizieren.

Der Zombie hat immer die gleiche Initiative wie der Zaubernde und ist immer direkt nach ihm im Kampf an der Reihe. Er hat 30 Lp, PA 0 und AU 7(+1xSteigerung). Er kann Nahkampfangriffe durchführen mit 2W8(+1xSteigerung) die 1W6 Schaden verursachen. Laufenwert: 5.

Man kann nie mehr als einen Untoten zum Leben erwecken.
\\ \hline
\end{tabular}
\\\\\\
\begin{tabular}{|p{\textwidth}|}
\hline
\begin{tabularx}{\textwidth}{X|X|X|X}
\textbf{Giftwolke} & Rang 2 & 2 Mana & SK: 7
\end{tabularx} \\ \hline
\begin{tabularx}{\textwidth}{X|X|X}
Standardaktion & Reichweite: 30 & Dauer: 5 Runden
\end{tabularx} \\ \hline
Der Zauber erzeugt eine Giftige Wolke in 30\,m Entfernung mit einem Radius von 20\,m. Jedes Lebewesen, welches sich zu Beginn seiner Runde in der Wolke befindet, muss eine Ausdauerprobe gegen den PW des Zaubers durchführen. Ist die Probe nicht erfolgreich erleidet das Lebewesen 2W6 Giftschaden.
\\ \hline
\end{tabular}
\\\\\\
\begin{tabular}{|p{\textwidth}|}
\hline
\begin{tabularx}{\textwidth}{X|X|X|X}
\textbf{Totendiener} & Rang 3 & 5 Mana & SK: 10
\end{tabularx} \\ \hline
\begin{tabularx}{\textwidth}{X|X|X}
30 Minuten & Reichweite: Berühren & Dauer: 2 Tage
\end{tabularx} \\ \hline
Eine Leiche die nicht älter als 10 Tage ist, wird für 2 Tage zum Leben erweckt. Der Untote hat 1 Lp und nimmt willenlos alle Befehle vom ZW entgegen.
Man kann nie mehr als einen Untoten zum Leben erwecken.
\\ \hline
\end{tabular}
\\\\\\
\begin{tabular}{|p{\textwidth}|}
\hline
\begin{tabularx}{\textwidth}{X|X|X|X}
\textbf{Elementar beschwören} & Rang 3 & 3 Mana pro Tag & SK: 9-15
\end{tabularx} \\ \hline
\begin{tabularx}{\textwidth}{X|X|X}
5 min & Reichweite: 20 & Dauer: Manakosten/3 Tage
\end{tabularx} \\ \hline
Der ZW kann mit einem Ritual einen Elementargeist (Wasser, Eis, Feuer, Humus, Erz oder Luft) beschwören. Die Schwierigkeit hängt von der Umgebung ab. Ist die Umgebung passend zu dem Elementargeist, dann ist die Beschwörung leichter. Die Fähigkeiten des Elementargeistes hängen vom Element ab und sind unten zu sehen.
\\ \hline
\end{tabular}

\begin{multicols}{2}

\begin{tcolorbox}[title={Feuerelementar},colbacktitle=red, coltitle=black]    
   \textbf{Lp:} 25
   \textbf{Mana:} 40
   \textbf{Init:} 1W8+2\\
   \textbf{Laufen:} 9
   \textbf{AU:} 13
   \textbf{RK:} 13\\
   
   \textbf{Feuerfaustangriff:} 2W8 Angriff, 2W10 Schaden, 1\,m Reichweite\\
   
   \textbf{Feuerball:} 1W8+1W10 ZP\\
   
   \textbf{Flammenmeer:} 1W8+1W10 ZP\\
   
   Nimmt halben Schaden durch physische Angriffe und doppelten von Angriffen mit Wasser.
\end{tcolorbox}

\begin{tcolorbox}[title={Wasserelementar},colbacktitle=blue, coltitle=white]    
   \textbf{Lp:} 50
   \textbf{Mana:} -
   \textbf{Init:} 1W8+2\\
   \textbf{Laufen:} 4 (an Land), 13 (im Wasser)
   \textbf{AU:} 12
   \textbf{RK:} 12\\
   
   \textbf{Wasserarm:} 1W8+1W10 Angriff, 2W10+3 Schaden, 5\,m Reichweite\\
   
   Nimmt halben Schaden durch physische Angriffe und doppelten von Angriffen mit Feuer.
\end{tcolorbox}

\begin{tcolorbox}[title={Eiselementar},colbacktitle=cyan, coltitle=black]    
   \textbf{Lp:} 40
   \textbf{Mana:} 40
   \textbf{Init:} 1W6+2\\
   \textbf{Laufen:} 7
   \textbf{AU:} 12
   \textbf{RK:} 15\\
   
   \textbf{Eisschlag:} 2W6+1 Angriff, 2W8+3 Schaden, 1\,m Reichweite\\
   
   \textbf{Eisgeschoss:} 1W8+1W10 ZP\\
   
   \textbf{Doppeltes Geschoss:} Kann den Zauber Eisgeschoss als Schnelle Aktion wirken, wenn er vorher als Standardaktion gewirkt wurde.
\end{tcolorbox}

\begin{tcolorbox}[title={Humuselementar},colbacktitle=green, coltitle=black]    
   \textbf{Lp:} 45
   \textbf{Mana:} 40
   \textbf{Init:} 1W6+2\\
   \textbf{Laufen:} 7
   \textbf{AU:} 13
   \textbf{RK:} 15\\
   
   \textbf{Rankenschlag:} 2W8+2 Angriff, 2W8+2 Schaden, 3\,m Reichweite\\
   
   \textbf{Wurzelfessel:} 1W8+1W10 ZP\\
   
   \textbf{Rankenangriff:} 1W8+1W10+2 ZP\\
   
   \textbf{Doppelte Ranke:} Immer wenn der Zauber Rankenangriff gewirkt wird entstehen 2 Ranken statt nur einer.
\end{tcolorbox}

\begin{tcolorbox}[title={Erzelementar},colbacktitle=brown, coltitle=white]    
   \textbf{Lp:} 80
   \textbf{Mana:} -
   \textbf{Init:} 1W4\\
   \textbf{Laufen:} 2
   \textbf{AU:} 4
   \textbf{RK:} 18\\
   
   \textbf{Steinschlag:} 2W8 Angriff, 4W12+2 Schaden, 1\,m Reichweite\\
\end{tcolorbox}

\begin{tcolorbox}[title={Luftelementar},colbacktitle=white, coltitle=black]    
   \textbf{Lp:} 20
   \textbf{Mana:} 50
   \textbf{Init:} 1W4\\
   \textbf{Laufen:} 2
   \textbf{AU:} 14
   \textbf{RK:} 14\\
   
   \textbf{Wirbelsturm:} 1W8+1W10 ZP\\
   
   \textbf{Kettenblitz:} 1W8+1W10+2 ZP\\
   
   \textbf{Starker Sturm:} Wenn Wirbelsturm gewirkt wird, werden alle numerischen Werte des Zaubers verdoppelt.
   
   Nimmt halben Schaden durch physische Angriffe und doppelten von magischen Angriffen.
\end{tcolorbox}

\end{multicols}

\section{Zerstörung}

Die Schule der Zerstörung befasst sich mit Zaubern, deren Ziel es ist Schaden anzurichten.
\\\\\\
\begin{tabular}{|p{\textwidth}|}
\hline
\begin{tabularx}{\textwidth}{X|X|X|X}
\textbf{Energiegeschoss} & Rang 1 & 2 Mana & SK: AU
\end{tabularx} \\ \hline
\begin{tabularx}{\textwidth}{X|X|X}
Standardaktion & Reichweite: 20/30/40 & Dauer: sofort
\end{tabularx} \\ \hline
ZW schießt eine Kugel magischer Energie auf ein Ziel. Verursacht 1W6 (plus 1W4 pro Steigerung) Schaden.
\\ \hline
\end{tabular}
\\\\\\
\begin{tabular}{|p{\textwidth}|}
\hline
\begin{tabularx}{\textwidth}{X|X|X|X}
\textbf{Flammenhand} & Rang 1 & 3 Mana & SK: 8
\end{tabularx} \\ \hline
\begin{tabularx}{\textwidth}{X|X|X}
Bonusaktion & Reichweite: Selbst & Dauer: 5 Runden
\end{tabularx} \\ \hline
ZW setzt seine Hand/Hände in Brand. Faustangriffe erhalten +1(+1xSteigerung) auf den Angriffswurf und verursachen 1W4 Extraschaden. Faustangriffe mit einer brennenden Hand werden gegen AU des Ziels geworfen.
\\ \hline
\end{tabular}
\\\\\\
\begin{tabular}{|p{\textwidth}|}
\hline
\begin{tabularx}{\textwidth}{X|X|X|X}
\textbf{Wirbelsturm} & Rang 1 & 2 Mana & SK: 7
\end{tabularx} \\ \hline
\begin{tabularx}{\textwidth}{X|X|X}
Standardaktion & Reichweite: 10 & Dauer: 1 Minute
\end{tabularx} \\ \hline
Zauber erzeugt einen Wirbelsturm in 10\,m Entfernung, der sich 5\,m pro Runde in eine Richtung fortbewegt. Jedes Lebewesen, in max 3\,m Entfernung zum Wirbelsturm, wirft eine Stemmen/Schieben Probe gegen den PW des Zaubers. Bei Nichtgelingen wird das Lebewesen 5\,m zurückgeschleudert. Lebewesen die sich in Richtung des Wirbelsturms bewegen sind verlangsamt. Als Bonusaktion kann die Bewegungsrichtung geändert werden.
\\ \hline
\end{tabular}
\\\\\\
\begin{tabular}{|p{\textwidth}|}
\hline
\begin{tabularx}{\textwidth}{X|X|X|X}
\textbf{Feuerball} & Rang 2 & 2 Mana & SK: AU
\end{tabularx} \\ \hline
\begin{tabularx}{\textwidth}{X|X|X}
Volle Aktion & Reichweite: 20/40/60 & Dauer: sofort
\end{tabularx} \\ \hline
Schießt einen Feuerball als Fernkampfangriff auf ein Ziel. Der Feuerball verursacht dabei Ges Schaden plus 1W6 pro Steigerung. Betroffen sind alle Lebewesen in einem Radius von 3\,m um den Einschlagspunkt.\\
\textbf{Ab PW 15:} Verbündete erleiden immer den minimalen Schaden.
\\ \hline
\end{tabular}
\\\\\\
\begin{tabular}{|p{\textwidth}|}
\hline
\begin{tabularx}{\textwidth}{X|X|X|X}
\textbf{Energieschwert} & Rang 2 & 3 Mana & SK: 8
\end{tabularx} \\ \hline
\begin{tabularx}{\textwidth}{X|X|X}
Standardaktion & Reichweite: 30 & Dauer: bis besiegt
\end{tabularx} \\ \hline
ZW beschwört ein magisches Schwert in max 30\,m Entfernung, dass sich relativ zu dem ZW mit bewegt. Das Schwert greift in Nahkampfreichweite mit dem PW der Zauberprobe an und verursacht 2W6 magischen Schaden (die Würfel werden für jede Steigerung verbessert bis hin zu 2W12 bei 3 Steigerungen). Das Schwert hat 15 Lp.
\\ \hline
\end{tabular}
\\\\\\
\begin{tabular}{|p{\textwidth}|}
\hline
\begin{tabularx}{\textwidth}{X|X|X|X}
\textbf{Kettenblitz} & Rang 2 & 3 Mana & SK: 10
\end{tabularx} \\ \hline
\begin{tabularx}{\textwidth}{X|X|X}
Standardaktion & Reichweite: 30 & Dauer: sofort
\end{tabularx} \\ \hline
ZW wirkt einen Blitz auf ein Ziel in 30\,m Entfernung der Ges+1W6 Schaden verursacht. Der Blitz springt auf das nächststehende Ziel über das max 7\,m entfernt ist. Der Blitz springt 2(+1xSteigerung) mal über. 
\\ \hline
\end{tabular}
\\\\\\
\begin{tabular}{|p{\textwidth}|}
\hline
\begin{tabularx}{\textwidth}{X|X|X|X}
\textbf{Eisgeschoss} & Rang 2 & 3 Mana & SK: RK
\end{tabularx} \\ \hline
\begin{tabularx}{\textwidth}{X|X|X}
Standardaktion & Reichweite: 20/40/60 & Dauer: sofort
\end{tabularx} \\ \hline
ZW lässt Eisgeschosse (beliebiger Form, handgroß) auf ein Ziel fliegen. Verursacht 2Ges(+2xSteigerung) Schaden.
\\ \hline
\end{tabular}
\\\\\\
\begin{tabular}{|p{\textwidth}|}
\hline
\begin{tabularx}{\textwidth}{X|X|X|X}
\textbf{Desintegration} & Rang 2 & 4 Mana & SK: AU
\end{tabularx} \\ \hline
\begin{tabularx}{\textwidth}{X|X|X}
Standardaktion & Reichweite: 40 & Dauer: sofort
\end{tabularx} \\ \hline
Zauber erzeugt einen 40\,m langen Kraftstrahl der Materialien auflöst. Auf einen Gegner gerichtet trifft der Strahl, wenn der PW größer oder gleich dem AU Wert des Ziels ist und verursacht Ges(+Ges*Steigerung) Schaden.
\\ \hline
\end{tabular}
\\\\\\
\begin{tabular}{|p{\textwidth}|}
\hline
\begin{tabularx}{\textwidth}{X|X|X|X}
\textbf{Erdbeben} & Rang 2 & 2 Mana/Runde & SK: 10
\end{tabularx} \\ \hline
\begin{tabularx}{\textwidth}{X|X|X}
Standardaktion & Reichweite: 120 & Dauer: sofort
\end{tabularx} \\ \hline
Der Zauber erzeugt in einem 30\,m Radius starke Schwingungen in dem Boden. Alle Lebewesen im Wirkungsbereich sind verlangsamt. Zu Beginn jeder Runde muss jedes Lebewesen eine Probe auf Akrobatik gegen den PW des Zaubers durchführen, ist diese nicht erfolgreich ist das Lebewesen niedergeschlagen.

Zu Beginn jeden Zuges des ZW öffnen sich zufällig 1W4 Erdspalten im Wirkungsgebiet. Diese Erdspalten sind 3\,m tief und 1W10\,m lang.

Gebäude im Wirkungsgebiet nehmen zu Beginn jeder Runde 3W10 Schaden.
\\ \hline
\end{tabular}
\\\\\\
\begin{tabular}{|p{\textwidth}|}
\hline
\begin{tabularx}{\textwidth}{X|X|X|X}
\textbf{Flammenmeer} & Rang 3 & 5 Mana & SK: 10
\end{tabularx} \\ \hline
\begin{tabularx}{\textwidth}{X|X|X}
Standardaktion & Reichweite: 20 & Dauer: 5 Runden
\end{tabularx} \\ \hline
ZW erzeugt ein Flammenmeer in max 20\,m Entfernung mit 5\,m  Radius. Jedes Lebewesen erhält, wenn es zu Beginn ihrer Runde im Bereich steht oder sich durch dieses hindurch bewegt, Ges+1W6 Magieschaden plus 1W10 pro Steigerung.
\\ \hline
\end{tabular}

\section{Veränderung}

Die Schule der Veränderung beinhaltet Zauber, die die Umwelt beeinflussen und verändern können.
\\\\\\
\begin{tabular}{|p{\textwidth}|}
\hline
\begin{tabularx}{\textwidth}{X|X|X|X}
\textbf{Pflanzenschub} & Rang 1 & 1 Mana/Runde & SK: 7
\end{tabularx} \\ \hline
\begin{tabularx}{\textwidth}{X|X|X}
Volle Aktion & Reichweite: 50 & Dauer: beliebig
\end{tabularx} \\ \hline
ZW lässt beliebig viele Pflanzen in bis zu 15 m Entfernung schnell anwachsen. Pflanzen wachsen mit bis zu 2\,m/Runde(+0,5xSteigerung).
\\ \hline
\end{tabular}
\\\\\\
\begin{tabular}{|p{\textwidth}|}
\hline
\begin{tabularx}{\textwidth}{X|X|X|X}
\textbf{Pfütze vertiefen} & Rang 1 & 1 Mana & SK: 6
\end{tabularx} \\ \hline
\begin{tabularx}{\textwidth}{X|X|X}
Bonusaktion & Reichweite: 30 & Dauer: sofort
\end{tabularx} \\ \hline
ZW kann eine existierende Pfütze um 1\,m (+1\,m\,x\,Steigerung) vertiefen. Mit bloßem Auge ist der Unterschied nicht wahrnehmbar.
\\ \hline
\end{tabular}
\\\\\\
\begin{tabular}{|p{\textwidth}|}
\hline
\begin{tabularx}{\textwidth}{X|X|X|X}
\textbf{Gefrieren} & Rang 1 & 2-4 Mana & SK: 7
\end{tabularx} \\ \hline
\begin{tabularx}{\textwidth}{X|X|X}
Standardaktion & Reichweite: Berühren & Dauer: sofort
\end{tabularx} \\ \hline
Lässt leblose wasserhaltige Objekte in Berührenreichweite gefrieren. Der Effekt des Gefrieren breitet sich max 5\,m pro bezahltem Mana in dem Objekt aus.
\\ \hline
\end{tabular}
\\\\\\
\begin{tabular}{|p{\textwidth}|}
\hline
\begin{tabularx}{\textwidth}{X|X|X|X}
\textbf{Elementare Manipulation} & Rang 1 & 1 Mana & SK: 7
\end{tabularx} \\ \hline
\begin{tabularx}{\textwidth}{X|X|X}
Standardaktion & Reichweite: 10 & Dauer: sofort
\end{tabularx} \\ \hline
Der ZW kann einfache Tricks mit den 4 Elementen Luft, Erde, Feuer und Wasser vollführen. Der SL entscheidet ob die Tricks im Rahmen des Zaubers sind. Beispiele sind:\\
Luft: Der ZW kann kleine Luftströme erzeugen um Kerzen auszublasen, Feuer anzufachen, Körperkühlen in Hitze.\\
Erde: ZW kann ein 10\,cm großen loch in der Erde erzeugen oder Sand aufwirbeln um einen Gegner zu behindern.
Feuer: Der ZW kann mit einem Fingerschnippen eine kleine Flamme (streichholzgroß) erzeugen, bestehendes Feuer vergrößern oder etwas anzünden.
Wasser: Der ZW kann ein Liter Wasser in Sichtweite frei bewegen (kein Wasser in einem Objekt oder Person).
\\ \hline
\end{tabular}
\\\\\\
\begin{tabular}{|p{\textwidth}|}
\hline
\begin{tabularx}{\textwidth}{X|X|X|X}
\textbf{Alarm} & Rang 1 & 1-2 Mana & SK: 8
\end{tabularx} \\ \hline
\begin{tabularx}{\textwidth}{X|X|X}
Standardaktion & Reichweite: 20 & Dauer: 5 Stunden/Mana
\end{tabularx} \\ \hline
ZW wird alarmiert wenn ein Lebewesen definierter Größe eine Grenze in bis zu 20\,m überschreitet.
\\ \hline
\end{tabular}
\\\\\\
\begin{tabular}{|p{\textwidth}|}
\hline
\begin{tabularx}{\textwidth}{X|X|X|X}
\textbf{Tonmanipulation} & Rang 1 & 1 Mana & SK: 6
\end{tabularx} \\ \hline
\begin{tabularx}{\textwidth}{X|X|X}
Bonusaktion & Reichweite: 10 & Dauer: 2 Minuten
\end{tabularx} \\ \hline
Mit dem Zauber können Töne wie Flüstergeräusche in bis zu 10\,m Entfernung erzeugt werden oder die eigene Stimme stark verstärkt werden. Der Effekt hält bis zu 1 Minute an.
\\ \hline
\end{tabular}
\\\\\\
\begin{tabular}{|p{\textwidth}|}
\hline
\begin{tabularx}{\textwidth}{X|X|X|X}
\textbf{Telekinese} & Rang 2 & 0.5 Mana/Runde & SK: Gewicht/kg
\end{tabularx} \\ \hline
\begin{tabularx}{\textwidth}{X|X|X}
Volle Aktion & Reichweite: 30 & Dauer: beliebig
\end{tabularx} \\ \hline
Lässt einen unbelebten Gegenstand schweben bis in bis zu PW m Entfernung. Die Schwierigkeit ist Gewicht/(2kg) des Gegenstandes. Solange der Gegenstand schwebt muss sich der ZW konzentrieren und kann keine andere Aktion durchführen.
\\ \hline
\end{tabular}
\\\\\\
\begin{tabular}{|p{\textwidth}|}
\hline
\begin{tabularx}{\textwidth}{X|X|X|X}
\textbf{Dunkelheit} & Rang 2 & 2 Mana & SK: 9
\end{tabularx} \\ \hline
\begin{tabularx}{\textwidth}{X|X|X}
2 Volle Aktionen & Reichweite: 20 & Dauer: 10\,min + 2\,min/Lvl
\end{tabularx} \\ \hline
Verdunkelt die Umgebung in 20\,m. Bei keiner Steigerung wird die Umgebung dämmrig, bei einer Steigerung nächtlich und bei zwei Steigerungen absolut dunkel. Der Effekt hält 10 Minuten + 2 Minute pro Spielerlevel.
\\ \hline
\end{tabular}
\\\\\\
\begin{tabular}{|p{\textwidth}|}
\hline
\begin{tabularx}{\textwidth}{X|X|X|X}
\textbf{Eisgegenstand} & Rang 2 & 3 Mana & SK: SL
\end{tabularx} \\ \hline
\begin{tabularx}{\textwidth}{X|X|X}
Volle Aktion & Reichweite: 5 & Dauer: sofort
\end{tabularx} \\ \hline
ZW kann einen Eisgegenstand aus Luftfeuchtigkeit(schwierig) oder Wasser(einfacher) erzeugen. Erschwernis wird vom Spielleiter bestimmt.
\\ \hline
\end{tabular}
\\\\\\
\begin{tabular}{|p{\textwidth}|}
\hline
\begin{tabularx}{\textwidth}{X|X|X|X}
\textbf{Extreme Temperatur} & Rang 2 & 5 Mana & SK: 9
\end{tabularx} \\ \hline
\begin{tabularx}{\textwidth}{X|X|X}
Volle Aktion & Reichweite: 100 & Dauer: 2 h
\end{tabularx} \\ \hline
ZW lässt die Temperatur um bis zu $10^\circ$C(+$40^\circ$CxSteigerung) in 100\,m Umkreis abfallen oder ansteigen.
\\ \hline
\end{tabular}
\\\\\\
\begin{tabular}{|p{\textwidth}|}
\hline
\begin{tabularx}{\textwidth}{X|X|X|X}
\textbf{Inhalt tauschen} & Rang 2 & 3 Mana  & SK: SL
\end{tabularx} \\ \hline
\begin{tabularx}{\textwidth}{X|X|X}
Standardaktion & Reichweite: 5 & Dauer: sofort
\end{tabularx} \\ \hline
Tauscht den Inhalt zweier Äußerlich identischen Gefäße in bis zu 5\,m Reichweite. 
\\ \hline
\end{tabular}
\\\\\\
\begin{tabular}{|p{\textwidth}|}
\hline
\begin{tabularx}{\textwidth}{X|X|X|X}
\textbf{Vergraben} & Rang 2 & 3 Mana & SK: 9
\end{tabularx} \\ \hline
\begin{tabularx}{\textwidth}{X|X|X}
Standardaktion & Reichweite: selbst & Dauer: 2 Runden
\end{tabularx} \\ \hline
Wenn der ZW auf Erdboden steht, kann er mit diesem verschmelzen und sich mit dem normalen Laufenwert unterirdisch bewegen. Der ZW kann sich so nur durch reinen Erdboden bewegen.
\\ \hline
\end{tabular}
\\\\\\
\begin{tabular}{|p{\textwidth}|}
\hline
\begin{tabularx}{\textwidth}{X|X|X|X}
\textbf{Regeneration} & Rang 2 & 5 Mana & SK: 10
\end{tabularx} \\ \hline
\begin{tabularx}{\textwidth}{X|X|X}
5 Volle Aktionen & Reichweite: Berührung & Dauer: 1 Erholungsphase
\end{tabularx} \\ \hline
Der Zauberwirker kann die Regeneration des Ziels beschleunigen. In der nächsten Regenerationsmöglichkeit des Ziels ist die Regeneration verdoppelt.
\\ \hline
\end{tabular}
\\\\\\
\begin{tabular}{|p{\textwidth}|}
\hline
\begin{tabularx}{\textwidth}{X|X|X|X}
\textbf{Magie bannen} & Rang 2 & 3/4/5 Mana & SK: GP Zauber
\end{tabularx} \\ \hline
\begin{tabularx}{\textwidth}{X|X|X}
Volle Aktion & Reichweite: 10/20/30 & Dauer: Sofort
\end{tabularx} \\ \hline
Bannt einen Zauber, wenn die Zauberprobe gegen den PW des zu bannenden Zauber gelingt. Auf mittlere Reichweite und hoher Reichweite erhöhen sich die Manakosten um jeweils 1 Mana.
\\ \hline
\end{tabular}
\\\\\\
\begin{tabular}{|p{\textwidth}|}
\hline
\begin{tabularx}{\textwidth}{X|X|X|X}
\textbf{Reinigung} & Rang 2 & 3 Mana & SK: SL
\end{tabularx} \\ \hline
\begin{tabularx}{\textwidth}{X|X|X}
Volle Aktion & Reichweite: Berühren & Dauer: Sofort
\end{tabularx} \\ \hline
Der Zauber kann Krankheiten und Flüche entfernen. Ist die Ursache solcher ein anderer Zauber gewesen ist die SK der PW jenen Zaubers. Ansonsten legt der Spielleiter die SK fest.
\\ \hline
\end{tabular}
\\\\\\
\begin{tabular}{|p{\textwidth}|}
\hline
\begin{tabularx}{\textwidth}{X|X|X|X}
\textbf{Stille} & Rang 2 & 2 Mana & SK: 8
\end{tabularx} \\ \hline
\begin{tabularx}{\textwidth}{X|X|X}
Standardaktion & Reichweite: 10\,m & Dauer: 10 Minuten
\end{tabularx} \\ \hline
Der ZW legt einen Punkt in maximal 10\,m Entfernung fest. In der Kugel mit Radius von 5\,m um den gewählten Punkt können keine Geräusche erzeugt werden.
\\ \hline
\end{tabular}
\\\\\\
\begin{tabular}{|p{\textwidth}|}
\hline
\begin{tabularx}{\textwidth}{X|X|X|X}
\textbf{Antimagiefeld} & Rang 3 & 5 Mana & SK: 12
\end{tabularx} \\ \hline
\begin{tabularx}{\textwidth}{X|X|X}
Standardaktion & Reichweite: selbst & Dauer: 1 Minute/Level
\end{tabularx} \\ \hline
In einem 5\,m Radius um den Zauberwirker können keine Zauber gewirkt werden und magische Gegenstände verlieren ihre Wirkung. Ein Zaubergeschoss, dass von außen in die Kugel eindringt, wird abgeschwächt, sodass Schaden den dieser Anrichten würde geviertelt wird.
\\ \hline
\end{tabular}
\\\\\\
\begin{tabular}{|p{\textwidth}|}
\hline
\begin{tabularx}{\textwidth}{X|X|X|X}
\textbf{Bärengestalt} & Rang 3 & 7 Mana & SK: 10
\end{tabularx} \\ \hline
\begin{tabularx}{\textwidth}{X|X|X}
Volle Aktion & Reichweite: selbst & Dauer: 1 Stunde
\end{tabularx} \\ \hline
ZW kann sich in einen Bären verwandeln. Durch die Verwandlung erhöht sich die Stärke und Konstitution um 1W6 (jede Steigerung erhöht den Würfel um eine Stufe bis hin zu 1W12 bei 3 Steigerungen). In Bärengestallt kann der ZW keine Zauber mehr wirken und alle bestehenden Zauber werden aufgelöst. Der Schaden des Faustangriffs wird um 2 erhöht.
\\ \hline
\end{tabular}

\section{Zauber überladen}

Mit der Probe Zauber überladen kann ein gewirkter Zauber verändert werden. Dazu wird zunächst ausgewürfelt, ob der Zauber gelingt. Ist das der Fall kann danach beschrieben werden, wie der Zauber modifiziert werden soll. Der Spielleiter legt je nach Beschreibung eine Schwierigkeit fest gegen die Zauber überladen geworfen werden soll. Ist das Zauber Überladen erfolgreich, geschieht der Modifizierung wie beschrieben. Bei einem Misserfolg hat der Zauber seine normale Wirkung, bei einer besonders schlechten Zauber überladen Probe können noch weitere ungeplante Nebenwirkungen auftreten.

\section{Erlernen von Magie}


\subsection{Erlernen von Magie durch Fertigkeitspunkte}

Zwischen den Sessions können Fp ausgegeben werden um neue Zauber zu erlernen. Die nötigen Fp entsprechen dem Level des Zaubers. Außerdem ist ein Mindestfertigkeitswürfel in der jeweiligen Zauberprobe (z.B. Verzauberungszauberprobe für Zauber aus der Verzauberungsschule). Level 1 Zauber können von jedem erlernt werden, während für Zauber vom Level 2 ein 1W6 und Level 2 nötig ist. Für Level 3 Zauber ist 1W10 und Level 5 die Voraussetzung.

{\rowcolors{2}{black!10!white!40}{black!50!white!50}
\begin{table}[h!]
\centering
\caption{Zaubervoraussetzungen}
\begin{tabular}{|cl|}
\hline
\textbf{Zauberlevel} & \textbf{Vorraussetzung}\\ \hline
1 & keine\\
2 & 1W6 in der entsprechenden Zauberprobe, Level 3\\
3 & 1W10 in der entsprechenden Zauberprobe, Level 5\\ \hline
\end{tabular}
\end{table}
}

\subsection{Erlernen der Magie durch Studium (Legacy)}
Der Prozess einen Zauber zu erlernen beansprucht Zeit in der Regel Zeit und eine Anleitung. Die Anleitung kann von einem magischen Buch stammen, in dem der Zauber zum lernen festgehalten ist, oder ein Lehrer, der den Zauber bereits beherrscht, kann den Charakter anleiten. Je höher der Rang desto mehr Zeit wird benötigt. Nachdem ein Charakter 1 Woche pro Rang des Zaubers sich mit einem Zauber beschäftigt hat, darf er eine entsprechende Zauberprobe werfen. Die Schwierigkeit ist für Zauber vom Rang 1 10, bei Rang 2 14 und bei Rang 3 18. Wenn der zu lernende Zauber einem bereits bekanntem Zauber ähnelt (SL entscheidet), kann die Probe um 1 erleichtert werden. Ist die Probe erfolgreich so beherrscht der Charakter den Zauber.

Ein Charakter kann auch versuchen einen Zauber den er gesehen hat oder von dem er gehört hat ohne Anleitung versuchen zu lernen. In diesem Fall verdoppelt sich die Zeitdauer und die Schwierigkeit wird um 2 erhöht. Ebenso kann der Charakter einen neuen Zauber entwickeln mit den gleichen Bedingungen. Entwickelt der Charakter einen neuen Zauber oder lernt einen Zauber ohne Anleitung, der einem bereits bekannten Zauber ähnelt, gelten die Regeln, als würde der Charakter unter Anleitung einen Zauber lernen, der keinem bekannten ähnelt.

Generell gilt, dass neu erstellte Charaktere auf Level 1 nur Zauber vom Rang 1 erlernen können.

\section{Alchemie}
Alchemie ist die Kunst magische Flüssigkeiten herzustellen. Alchemie hat seine eigene Probe und funktioniert ähnliche zum Zaubern. Der Charakter kann Rezepte auf die gleiche Art lernen wie Zauber gelernt werden, statt dann die Zauber zu beherrschen ist der Alchemist in der Lage den entsprechenden Zauber als Trank, Bombe oder ähnliches herzustellen. Wie genau der Trank auf Basis eines Zaubers funktioniert wird vom Spieler genau beschrieben und vom Spielleiter genehmigt. Zum Brauen eines Trankes wird immer auf Alchemie geworfen. Die Schwierigkeit hängt vom Zauber ab der gebraut wird. Stufe 1 entspricht SK 7, Stufe 2 entspricht SK 9 und Stufe 3 entspricht SK 11. Das Brauen eines Zaubers dauert 10\,min pro Stufe des Zaubers. Zum Brauen werden Zutaten benötigt, die Schwierigkeit diese zu beschaffen hängt von der Umgebung ab und wird vom Spielleiter bestimmt. Realistische Zeiten um die Zutaten für einen Trank zu sammeln sind 2 Stunden bis 2 Tage. Zutaten können gekauft werden oder in der Natur selbst gesammelt werden. Der Spieler kann maximal 4 fertig gebraute Tränke bei sich tragen. Mit jedem Charakterlevel wird das Maximum um 1 erhöht.

\end{document}